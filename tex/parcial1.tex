\section{Paradoja de Russel.}
\textbf{Conceptos Primarios:} Conceptos no definidos los cuales se asumen debido a su claridad.\\

Sea $\mathcal{N}$ el conjunto de todos los conjuntos, entonces $\mathcal{N} \in \mathcal{N}$. Ahora, considere el conjunto $\mathcal{A}$ de todos los conjuntos que no son elementos de sí mismos. 

\[ \mathcal{N} \notin \mathcal{A} \]

Así $\mathcal{A} \in \mathcal{A} \vee \mathcal{A}\notin\mathcal{A}$ es una tautología, y de hecho se cumple una única de las proposiciones.\\

Si $\mathcal{A} \in \mathcal{A}$ es verídica, entonces $\mathcal{A} \notin \mathcal{A}$. Se tendría así $\mathcal{A} \in \mathcal{A} \wedge	\mathcal{A} \notin \mathcal{A}$. Por lo tanto $\mathcal{A} \in \mathcal{A}$ no es verídica.\\

Si $\mathcal{A} \notin \mathcal{A}$ es verídica, entonces $\mathcal{A} \in \mathcal{A}$, se tendría así  $\mathcal{A} \notin \mathcal{A} \wedge \mathcal{A} \in \mathcal{A}$. Por lo tanto  $\mathcal{A} \notin \mathcal{A}$ no es verídica.En ambos casos se ha obtenido una contradicción.\\
\medskip
\\
\textbf{Proposición 0.} No existe una biyección entre $\mathbb{N}$ y $\mathbb{R}$.\\
\textbf{Demostración.}\\
Suponga que existe una función $f:\mathbb{N}\longrightarrow\mathbb{R}$, para cada $n \in \mathbb{N}$ sea
$A_{n}.a_{n1}a_{n2}...$ la expresión decimal de la imagen de $n$ bajo $f$. Así $a_{nj}$ es el decimal j-ésimo de $f(n).$ Considere el número real z cuya expresión decimal es $0.z_{1}z_{2}...$ siendo $z_i \in \{1,2\}$ y además

\[z_i = \left \{ \begin{matrix} 1 & \mbox{si }a_{ii}\neq1.
\\ 2 & \mbox{si } a_{ii}=1\end{matrix}\right. \]

Suponga que para algún $n \in \mathbb{N}$ $z=f(n)$, luego:

\[z_n = \left \{ \begin{matrix} 1 & \mbox{si }a_{nn}\neq1.
\\ 2 & \mbox{si } a_{nn}=1\end{matrix}\right. \]

Si $z_n = 1$, entonces $a_{nn} \neq 1$ y así la identidad $z = f(n)$ implica que $a_{nj}$. Sea $n \in \mathbb{N}$.
\begin{iteritem}
\item i) Si $z_n = 1$, entonces $a_{nn} \neq 1$ existen dos posibilidades, para que $z = f(n)$ la primera es $a_{nn} = 0$
y la segunda $a_{nn} = 2$. La primera implica que $a_{nj} = 9$ $\forall j > n$ $\wedge$ $z_j \notin \{1,2\}$ $\forall j \in \mathbb{N}$.
La segunda posibilidad i.e. $a_{nn} = 2$ implicaría que $a_{nj} = 0$ $\forall j > n$ $\wedge$ $z_j = 9$ $\forall j > n$ y esto último 
no es posible.
\item ii) Si $z_n = 2$ entonces $a_{nn} = 1$ la identidad $z = f(n)$ implica $a_{nj} = 9$ $\forall j > n$ $\wedge$ $z_j = 0$ 
$\forall j > n$ lo cual contradice que $z_j = \{1,2\}$ $\forall j \in \mathbb{N}$. 
\end{iteritem}

\section{Axiomas de Zermelo-Fraenkel}

Aprincipios de l siglo XX, fue el matemático aleman \textbf{Ernest Zermelo} quien puso la teoría de conjuntos sobre una base
aceptable reduciendola a un sistema axiomático más restringido que no permita la obtención de la \textbf{Paradoja de Russel}.
Las ideas de \textbf{Zermelo} fueron después precisadas por \textbf{Thoralf Skolen} y \textbf{Abrahm Fraenkel}, resultado de
ello la primera teoría axiomática de conjuntos, conocida como \textbf{teoría de Zermelo-Fraenkel}, aunque sería más adecuado 
llamarla \textbf{teoría de Zermelo-Fraenkel-Skolen}.\\

Hasta ahora el alumno ha desarrollado una noción de \textit{conjunto}, y se ha
hablado y trabajado con conjuntos como $\mathbb{N}$, $\mathbb{R}$, \O,
\textit{etc}... En éste curso se busca presentar una fundación rigurosa de las
matemáticas conocidas en términos de la teoría axiomática de conjuntos. Los
axiomas de Zermelo-Fraenkel conforman, junto con el Axioma de Elección, el
modelo de facto para la teoría de conjuntos.
En las notas, las letras se referirán a conjuntos a menos que se especifique. El
orden de presentación de los axiomas es primeramente la noción de igualdad,
luego axiomas de construcción de conjuntos, axiomas de existencia de ciertos
conjuntos, y posteriormente en las notas, el axioma de elección.:

\begin{axiom} \textbf{Axioma de Extensionalidad}\\
Los conjuntos son iguales si tienen los mismos elementos.
    \[
        \forall A, \forall B \quad A = B \, \Longleftrightarrow (\forall \mathbf{x};  \quad \mathbf{x} \in A \, \Longleftrightarrow
        \, \mathbf{x} \in B)
    \]
\end{axiom}
\begin{axiom} \textbf{Axioma del Vacío.}\\
Existe un conjunto sin elementos.
    \[
    \exists B \, \quad \forall \mathbf{x}; \quad \mathbf{x} \notin B
    \]

\end{axiom}

\begin{axiom} \textbf{Axioma de Parejas.}\\
    Para cualquiera dos conjuntos, existe un tercer conjunto del cual son
    elementos.
   \[
        \forall A, \forall B \quad \exists C  \quad (\forall \mathbf{x}; \quad \mathbf{x} \in B \Longleftrightarrow \mathbf{x} = A \vee \mathbf{x}=B)
    \]
\end{axiom}

\begin{axiom} \textbf{Axioma de la unión.}\\
    Para cualquier conjunto, existe un conjunto cuyos elementos son todos los
    elementos de los miembros del conjunto.
    \[
        \forall A, \forall B \quad (\forall \mathbf{x}; \quad \mathbf{x}\in B \Longleftrightarrow \exists \mathbf{y} \in A, \mathbf{x} \in \mathbf{y})
    \]
\end{axiom}
\begin{axiom} \textbf{Axioma del Conjunto Potencia.}\\
    Para cada conjunto existe otro cuyos elementos cumplen la propiedad necesaria y suficiente de 
    que en sus elementos están elementos del ya dado.
    \[
        \forall A \quad \exists B \quad [\forall \mathbf{y}; \quad \mathbf{y} \in B \Longleftrightarrow ( \mathbf{x} \in \mathbf{y} \Longleftrightarrow \mathbf{x} \in A)]
    \]
\end{axiom}
\begin{axiom} \textbf{Esquema Axiomático de Especificación.}\\
    Dado un conjunto y una proiedad independiente de tal conjunto, existe otro cuyos elementos son tales que 
    cumplen tal propiedad. Sea $A$ un conjunto y $\mathcal{P}(\mathbf{x})$ una propiedad independiente de $A$.
    \[
        \forall A \exists B \quad [\forall \mathbf{x}; \quad \mathbf{x} \in B \Longleftrightarrow \mathbf{x} \in A \wedge \mathcal{P}(\mathbf{x}) ]
   \]
\end{axiom}
El axioma anterior presenta una restricción importante, sólamente se pueden
co nstruir conjuntos a partir de conjuntos cuya existencia es previamente
conocida. \\

\begin{proposition}
\textbf{Paradoja de Russell:} No existe un conjunto al que pertenezca todo conjunto.
\end{proposition}
\begin{proof2}
Sea $\mathcal{A}$ un conjunto, para probar la proposición basta demostrar que existe un conjunto que no pertenezca a $\mathcal{A}$.
Considere la propiedad siguiente: 
\[
\mathcal{P}(\mathbf{x}) \quad | \quad \mathbf{x} \notin \mathbf{x}
\]
En virtud de \textbf{ZF 6} implica que existe un conjunto $\mathcal{B}$ tal que:
\[
\forall \mathbf{x}, \quad \mathbf{x} \in \mathcal{B} \Longleftrightarrow \mathbf{x} \in \mathcal{A} \wedge \mathcal{x} \notin \mathbf{x}
\]
Se verificará que $\mathcal{B} \notin \mathcal{A}$. En efecto, procediendo por \textit{Reductio ad absurdum}, suponga que $\mathcal{B} \in \mathcal{A}$, se tiene que:
\[ 
\mathcal{B} \in \mathcal{B} \vee \mathcal{B} \notin \mathcal{B}
\]
\begin{iteritem}
\item Si $\mathcal{B} \in \mathcal{B} \Longrightarrow \mathcal{B} \notin \mathcal{B}$.
\item Si $\mathcal{B} \notin \mathcal{B} \Longrightarrow \mathcal{B} \in \mathcal{B}$.
\end{iteritem}

En cualquier caso se tiene una contradicción. Por lo tanto $\mathcal{B} \notin \mathcal{A}$.\\
\end{proof2}
\\
\\
\medskip
\\
\\
\medskip
\\
\\
\medskip

\begin{centering}
	\textbf{Definiciones Iniciales.} \\
\end{centering}
Los axiomas anteriores justifican las siguientes definiciones en el sentido en que se asegura
la existencia de los conjuntos que se definen.

\begin{iteritem}
\item \textbf{I)} Para cada par de conjuntos $\mathcal{A},\mathcal{B}$ se define el conjunto determinado por ellos, 
como el conjuntoo cuyos únicos elementos son $\mathcal{A},\mathcal{B}$ ta conjunto se denotará por 
$\{\mathcal{A},\mathcal{B}\}$ y su existenica es asegurada por \textbf{ZF 3}.
\item \textbf{II)} El conjunto definido por \textbf{ZF 4}, se denotará por $\bigcup \mathcal{A}$ y se denominará
la unión de elementos de $\mathcal{A}$, o bien se denotará por $\bigcup_{\mathbf{x} \in \mathcal{A}} \mathbf{x}$.
\item \textbf{III)} El conjunto definido por \textbf{ZF 4} se denomina \textbf{conjunto potencia}
de $\mathcal{A}$, siendo $\mathcal{A}$ un conjunto, y se denotará por $\mathcal{P}(\mathcal{A})$.
\item \textbf{IV)} Sean $\mathcal{A},\mathcal{B}$ conjuntos, sea $\mathcal{P}(\mathbf{x})$ la propiedad $\mathbf{x} \in \mathcal{B}$.
Por el \textbf{ZF 6} para el conjunto $\mathcal{A}$.
\[ \exists \mathcal{D} \quad (\forall \mathbf{x}; \quad \mathbf{x} \in \mathcal{D} \Longleftrightarrow \mathbf{x} \in \mathcal{A} \wedge \mathbf{x} \in \mathcal{B})\]
$\mathcal{D}$ se denomina \textbf{intersección} de $\mathcal{A}$ y $\mathcal{B}$ y se denota por $\mathcal{A} \bigcap \mathcal{B}$.
\item \textbf{V)} Sean $\mathcal{A}, \mathcal{B}$ conjuntos, sea $\mathcal{P}(\mathbf{x})$ la propiedad $\mathbf{x} \notin \mathcal{B}$.
Por \textbf{ZF 6}, para el conjunto $\mathcal{A}$.
\[
\exists \mathcal{D} \quad (\forall \mathbf{x}; \quad \mathbf{x} \in \mathcal{A} \Longleftrightarrow \mathbf{x} \in \mathcal{A} \wedge \mathbf{x} \notin \mathcal{B})
\]
$\mathcal{D}$ se denomina la diferencia de $\mathcal{A}$ con $\mathcal{B}$, o el complemento de $\mathcal{B}$ en $\mathcal{A}$,
y se denota como $\mathcal{A} \setminus \mathcal{B}$.
\item \textbf{VI)} Sean $\mathcal{A}, \mathcal{B}$ conjuntos, por el axioma de parejas existe un conjunto cuyoos únicos elementos son
$\mathcal{A}$ y $\mathcal{B}$, sea $\mathcal{C}$ tal conjunto, i.e. $\mathcal{C} = \{ \mathcal{A}, \mathcal{B}\}$. Por \textbf{ZF 4}
existe un conjunto $\mathcal{D}$ tal que:
\[
\exists \mathcal{D} \quad (\forall \mathbf{x}; \quad \mathbf{x} \in \mathcal{D} \Longleftrightarrow \exists \mathbf{y} \in \mathcal{C}, \mathbf{x} \in \mathbf{y})
\]
Es decir tenemos que $\mathcal{D} = \bigcup_{\textbf{x} \in \mathcal{C}} \mathbf{x}$, \mathcal{D} se denomina la unión de $\mathcal{A}$
y $\mathcal{B}$, y se denotará por $\mathcal{A} \bigcup \mathcal{B}$. El axioma \textbf{ZF 6} garantiza la unicidad de los conjuntos así definidos.
\item \textbf{VII)} Sea $\mathcal{A}$ un conjunto distinto del vacío. Por el axioma \textbf{ZF 6} existe un conjunto $\mathcal{B}$, tal
que
\[
\exists \mathcal{B} \quad [ \forall \mathbf{x}; \quad \mathbf{x} \in \mathcal{B} \Longleftrightarrow (\forall \mathbf{y}; \quad \mathbf{y} \in \mathal{A} \Longrightarrow \mathbf{x} \in \mathbf{y})]
\]
$\mathcal{B}$ se denomna la intersección de $\mathcal{A}$, se denotará por $\bigcap \mathcal{A}$ o bien $\bigcap_{\mathbf{x} \in \mathcal{A} \mathbf{x}}$.
\item \textbf{VIII)} El conjunto cuya existencia se justifica por \textbf{ZF 2}, es decir el conjunto sin elementos se denotará por \O.
\end{iteritem}

\begin{proposition} 
\textbf{Unicidad del Vacío:} Hay un único conjunto que no tiene elementos.\\
\underline{Demostración.} (La demostración es elemental)
\end{proposition}

\begin{proposition}
	Sean $\mathcal{A,B,C}$ conjuntos. Entonces se cumplen las siguientes propiedades.
	\begin{iteritem}
		\begin{centering}
	\item \textbf{Leyes Distributivas.}
		\begin{iteritem}			
			
		\item $\mathcal{A} \bigcap (\mathcal{B} \bigcup \mathcal{C}) = (\mathcal{A} \bigcap \mathcal{B}) \bigcup (\mathcal{A} \bigcap \mathcal{C}).$
		\item $\mathcal{A} \bigcup (\mathcal{B} \bigcap \mathcal{C}) = (\mathcal{A} \bigcup \mathcal{B}) \bigcap (\mathcal{A} \bigcup \mathcal{C}).$
		\end{iteritem}

	\item \textbf{Leyes de DeMorgan.}
		\begin{iteritem}	
		\item $\mathcal{C}$\textbackslash$(\mathcal{A} \bigcup \mathcal{B}$) = $(\mathcal{C}$\textbackslash$\mathcal{A}) \bigcap (\mathcal{C}$ \textbackslash $\mathcal{B}).$	
		\item $\mathcal{C}$\textbackslash$(\mathcal{A} \bigcap \mathcal{B}$) = $(\mathcal{C}$\textbackslash$\mathcal{A}) \bigcup (\mathcal{C}$ \textbackslash $\mathcal{B}).$	
		\end{iteritem}

	\item \textbf{Leyes Asociativas.}
		\begin{iteritem}
		\item $\mathcal{A} \bigcup (\mathcal{B} \bigcup \mathcal{C}) = (\mathcal{A} \bigcup \mathcal{B}) \bigcup \mathcal{C}).$
		\item $\mathcal{A} \bigcap (\mathcal{B} \bigcap \mathcal{C}) = (\mathcal{A} \bigcap \mathcal{B}) \bigcap \mathcal{C}).$
		\end{iteritem}

	\item \textbf{Leyes de Idempotencia.}
		\begin{iteritem}
		\item $\mathcal{A} \bigcup \mathcal{A} = \mathcal{A}.$
		\item $\mathcal{A} \bigcap \mathcal{A} = \mathcal{A}.$
		\end{iteritem}
		
		\end{centering}
	\end{iteritem}
\end{proposition}
\begin{proof2}
Se verificará que  $\mathcal{A} \cap (\mathcal{B} \cup \mathcal{C}) = (\mathcal{A} \cap \mathcal{B}) \cup (\mathcal{A} \cap \mathcal{C})$. Para probar la identidad anterior,
basta verificar en virtud del axioma de extensionalidad que los conjuntos  $\mathcal{A} \cap (\mathcal{B} \cup \mathcal{C})$ y $(\mathcal{A} \cap \mathcal{B}) \cup (\mathcal{A} \cap \mathcal{C})$ tienen los mismos elementos. \\ 

Dado $\mathbf{x}, \mathbf{x} \in \mathcal{A} \cap (\mathcal{B} \cup \mathcal{C})$ es equivalente a cada una de las siguientes afirmaciones siguientes: 
	\[ \mathbf{x} \in \mathcal{A} \quad \wedge \quad \mathbf{x} \in \mathcal{B} \cup \mathcal{C} \]
	\[ \mathbf{x} \in \mathcal{A} \quad \wedge \quad [ \mathbf{x}\in\mathcal{B} \vee \mathbf{x}\in\mathcal{C}  ]\]
	\[ [\mathbf{x}\in\mathcal{A} \wedge \mathbf{x}\in\mathcal{B}] \quad \vee [\mathbf{x}\in\mathcal{A} \wedge \mathbf{x}\in\mathcal{C}]\]
	\[ \mathbf{x}\in\mathcal{A}\cap\mathcal{B} \quad \vee \quad \mathbf{x}\in\mathcal{A}\cap\mathcal{C}\]
	\[ \mathbf{x} \in (\mathcal{A}\cap\mathcal{B})\cup(\mathcal{A}\cap\mathcal{C})\]

Se verifcará ahora $\mathcal{C}$\textbackslash$(\mathcal{A}\cup\mathcal{B}) = (\mathcal{C}$\textbackslash$\mathcal{A})\cap(\mathcal{C}$\textbackslash$\mathcal{B})$. Dado $\mathbf{z}, \mathbf{z}\in\mathcal{C}$\textbackslash$(\mathcal{A}\cup\mathcal{B})$ es equivalente a cada una de las siguientes proposiciones: 

\[ \mathbf{z}\in\mathcal{C} \quad \wedge \quad \mathbf{z}\notin(\mathcal{A}\cup\mathcal{B}), \quad \mathbf{z}\in\mathcal{C} \quad \wedge \quad \neg[\mathbf{z}\in \mathcal{A}\cup\mathcal{B}]\]
\[\mathbf{z}\in\mathcal{C} \quad \wedge \quad \neg[\mathbf{z}\in\mathcal{A} \vee \mathbf{z}\in\mathcal{B}], \quad \mathbf{z}\in\mathcal{C} \quad \wedge\quad [\mathbf{z}\notin\mathcal{A} \wedge \mathbf{z}\notin\mathcal{B}]\]
\[\mathbf{z}\in\mathcal{C} \quad \wedge \quad [\mathbf{z}\notin\mathcal{A} \wedge \mathbf{z}\notin\mathcal{B}], \quad [\mathbf{z}\in\mathcal{C} \wedge \mathbf{z}\notin\mathcal{A}] \quad \wedge \quad [\mathbf{z}\in\mathcal{C}\wedge\mathbf{z}\notin\mathcal{B}]\] 
\begin{centering}
	$\mathbf{z}\in(\mathcal{C}$\textbackslash$\mathcal{A})\quad\wedge\quad\mathbf{z}\in(\mathcal{C}$\textbackslash$\mathcal{B}), \quad \mathbf{z}\in(\mathcal{C}$\textbackslash$\mathcal{A})\cap(\mathcal{C}$\textbackslash$\mathcal{B})$\\
\end{centering}
$\therefore$ el axioma de extensionalidad permite concluir la identidad $\mathcal{C}$\textbackslash$(\mathcal{A} \bigcup \mathcal{B}$) = $(\mathcal{C}$\textbackslash$\mathcal{A}) \bigcap (\mathcal{C}$ \textbackslash $\mathcal{B})$. Las demás propiedades se demuestran de manera análoga. 
\end{proof2} 

\textbf{Definición:}\\
Sean $\mathcal{A,B}$ conjuntos se dice que $\mathcal{A}$ es subconjunto de $\mathcal{B}$ si $\forall \mathbf{x}(\mathbf{x}\in\mathcal{A} \Longrightarrow \mathbf{x}\in\mathcal{B})$. Ello se denotará por $\mathcal{A}\subset\mathcal{B}$.\\

\begin{proposition}
Sean $\mathcal{A,B,C,D}$ conjuntos. Entonces se cumplen las siguientes propiedades. 
	\begin{iteritem}
	\item I) $\mathcal{A}\cap\mathcal{B} \subset \mathcal{A} \quad \wedge \quad \mathcal{A} \subset \mathcal{A}\cup\mathcal{B}$
	\item II) Si $\mathcal{A}\subset\mathcal{C}\quad\wedge\quad\mathcal{B}\subset\mathcal{D}$,$\quad$ entonces $\mathcal{A}\cap\mathcal{B}\subset\mathcal{C}\cap\mathcal{D}\quad\wedge\quad\mathcal{A}\cup\mathcal{B}\subset\mathcal{C}\cup\mathcal{D}$.
	\end{iteritem}
\end{proposition}
\begin{proof2}
	\begin{iteritem}
	\item I) Sea $\mathbf{x}\in\mathcal{A}\cap\mathcal{B}$, entonces $\mathbf{x}\in\mathcal{A} \quad\wedge\quad \mathbf{x}\in\mathcal{B}$.
		$\therefore\quad\forall\mathbf{x}(\mathbf{x}\in\mathcal{A}\cap\mathcal{B}\longrightarrow\mathbf{x}\mathcal{A})$ i.e $\mathcal{A}\cap\mathcal{B}\subset\mathcal{A}$. Sea $\mathbf{y}\in\mathcal{A}$, entonces $\mathbf{y}\in\mathcal{A}\quad\vee\quad\mathbf{y}\in\mathcal{B}$, i.e. $\mathbf{y}\in\mathcal{A}\cup\mathcal{B} \quad\therefore\quad\forall\mathbf{y}(\mathbf{y}\in\mathcal{A}\longrightarrow\mathbf{y}\mathcal{A}\cup\mathcal{B})$ i.e. $\mathcal{A}\subset\mathcal{A}\cup\mathcal{B}$. 
	\item II) Sea $\mathbf{x}\in\mathcal{A}\cup\mathcal{B}$, entonces $\mathbf{x}\in\mathcal{A}\quad\vee\quad\mathbf{x}\in\mathcal{B}$. Por hipótesis se tiene:
\[\mathbf{x}\in\mathcal{A}\longrightarrow\mathbf{x}\in\mathcal{C},\quad\mathbf{x}\in\mathcal{B}\longrightarrow\mathbf{x}\in\mathcal{D}\]
		Por el silogismo disyuntivo se concluye que $\mathbf{x}\in\mathcal{C}\quad\vee\quad\mathbf{x}\in\mathcal{D}$ i.e. $\mathbf{x}\in\mathcal{C}\cap\mathcal{D}$. Se ha verificado que $\forall\mathbf{x}[\mathbf{x}\in(\mathcal{A}\cup\mathcal{B})\longrightarrow\mathbf{x}\in(\mathcal{C}\cup\mathcal{D})]$ $\thereore \quad\mathcal{A}\cup\mathcal{B}\subset\mathcal{C}\cup\mathcal{D}$. \\

Sea $\mathbf{y}\in\mathcal{A}\cap\mathcal{B}$, entonces $\mathbf{y}\in\mathcal{A}\quad\wedge\quad\mathcal{B}\quad\therefore\quad\mathbf{y}\in\mathcal{C}\quad\wedge\quad\mathbf{y}\in\mathcal{D}$, i.e. $\mathbf{y}\in\mathcal{C}\cap\mathcal{D}$.
\end{iteritem}
\end{proof2}

\begin{proposition}
Sean $\mathcal{A,B}$ conjuntos. Son equivalentes las siguientes proposiciones:
	\begin{iteritem}
	\item $\mathcal{A}\subset\mathcal{B}$.
	\item $\mathcal{A}=\mathcal{A}\cap\mathcal{B}$.
	\item $\mathcal{B}=\mathcal{A}\cup\mathcal{B}$.
	\end{iteritem}
\end{proposition}
\begin{proof2}
I)$\Longrightarrow$ II) Desde que $\mathcal{A}\cap\mathcal{A}=\mathcal{A}$ (Ley de Idempotencia). Por la \textbf{Proposición x-II)}, dado que $\mathcal{A}\subset\mathcal{A}$ $\wedge$ $\mathcal{A}\subset\mathcal{B}$. Se tiene que $\mathcal{A}\cap\mathcal{A}\subset\mathcal{A}\cap\mathcal{B}$ i.e $\mathcal{A}\subset\mathcal{A}\cap\mathcal{B}$ y por la \textbf{Proposición x-I)}, $\mathcal{A}\cap\mathcal{B}\subset\mathcal{A}$. $\therefore \quad \forall\mathbf{x}(\mathbf{x}\in\mathcal{A}\Longleftringhtarrowthcal\mathbf{x}\in\mathcal{A}\cap\mathcal{B})$. El axioma de extensionalidad implica que $\mathcal{A}=\mathcal{A}\cap\mathcal{B}$.\\

II)$\Longrightarrow$III) Por la \textbf{Proposición x-I)} $\mathcal{B}\subset\mathcal{A}\cup\mathcal{B}$. Por hipótesis $\mathcal{A}=\mathcal{A}\cap\mathcal{B}$.\\
	$\therefore\quad\mathcal{A}\cup\mathcal{B}=(\mathcal{A}\cap\mathcal{B})\cup\mathcal{B}=(\mathcal{A}\cup\mathcal{B})\cap(\mathcal{B}\cup\mathcal{B})=(\mathcal{A}\cup\mathcal{B})\cap\mathcal{B}$ (Ley Distributiva y de Idempotencia).\\

Además $(\mathcal{A}\cup\mathcal{B})\cap\mathcal{B}\subset\mathcal{B}$ por la \textbf{Proposición X-I)} $\therefore$ $\mathcal{A}\cup\mathcal{B}\subset\mathcal{B}$. Como $\mathcal{B}\subset\mathcal{A}\cup\mathcal{B}\quad\wedge\quad\mathcal{A}\cup\mathcal{B}\subset\mathcal{B}$ $\forall\mathbf{x}(\mathbf{x}\in\mathcal{B} \Longleftrightarrow \mathbf{x}\in\mathcal{A}\cup\mathcal{B})$. Por el axioma de extensionalidad, se concluye que $\mathcal{B}=\mathcal{A}\cup\mathcal{B}$.\\

	III)$\Longrightarrow$I) Por hipótesis $\mathcal{B}=\mathcal{A}\cup\mathcal{B}$, $\mathcal{A}\subset\mathcal{A}\cup\mathcal{B}$ \textbf{Proposición X-I)} $\therefore$ $\mathcal{A}\subset\mathcal{B}$.\\
\end{proof2}
\newpage
%%%%%%%%%%%%%%%%%%%%%%%%%%%%%%%%%%%%%%%%%%%%%%%%%%%%%%%%%%%%%%%%%%%%%%%%%%%%%%%%%%%%%%%%%%%%%%%%%%%%%%%%%%%%%%%%%%%%%%%%%%%%%%%%%%%%%%%%%%%%%%%%%%%%%%%%%%%%%%%%%%%%%%%%
\textbf{Ejercicios 1:} Sean $\mathcal{A,B}$ subconjuntos de $\mathcal{D}$. Demuestre:
\begin{iteritem}
\item   I) $\mathcal{A}\cap(\mathcal{D}$\textbackslash$\mathcal{A})=$ \O
\item  II) $\mathcal{D}$\textbackslash$(\mathcal{D}$\textbackslash$\mathcal{A})=\mathcal{A}$
\item III) $\mathcal{A}\subset\mathcal{B}$, si y sólo sí, $(\mathcal{D}$\textbackslash$\mathcal{B})\subset(\mathcal{D}$\textbackslash$\mathcal{A})$\\
\end{iteritem}

\textbf{Definición:}\\ 
Sean $\mathbf{a,b}$ conjuntos, elementos de un conjunto $\mathcal{A}$, se define la pareja ordenada determinada por $\mathbf{a,b}$ y denotada por $(\mathbf{a,b})$ como el conjunto $\{\{\mathbf{a},\{\mathbf{a,b}\}\}$.\\

Observe que $\{\mathbf{a}\}$ es un conjunto por el axioma de especificación, ya que por tal axioma existe un conjunto $\mathcal{B}$ tal que:
\[\forall\mathbf{x}\quad[\mathbf{x}\in\mathcal{B} \Longleftrightarrow \mathbf{x}\in\mathcal{A} \wedge \mathcal{P}(\mathbf{x}) ]\] 
Siendo $\mathcal{P}(\mathbf{x})$ la propiedad $\mathbf{x}=\mathbf{a}$. Por otra parte $\{\mathbf{a,b}\}$ es un conjunto en virtud del axioma de parejas. Este mismo axioma justifica que $\{\{\mathbf{a}\},\{\mathbf{a,b}\}\}$ es un conjunto. \\

\begin{proposition}
	\\
	Sean $\mathbf{a,b,c,d}$ elementos de un conjunto $\mathcal{A}$, entonces:
	\begin{center}
	$(\mathbf{a},\mathbf{b})=(\mathbf{c},\mathbf{d})$, si y sólo si, $\mathbf{a}=\mathbf{c}$ $\wedge$ $\mathbf{b}=\mathbf{d}$.
	\end{center}
\end{proposition}

\begin{proof2}
$\Longleftarrow)$ Si $\mathbf{a}=\mathbf{c}$ $\wedge$ $\mathbf{b}=\mathbf{d}$, entonces
\[ \{\{\mathbf{a} \}, \{ \mathbf{a}, \mathbf{b} \} \} = \{ \{ \mathbf{c} \}, \{\mathbf{c},\mathbf{d} \} \},\quad i.e. \quad (\mathbf{a},\mathbf{b})=(\mathbf{c},\mathbf{d}) \]
$\Longrightarrow$) Por hipótesis $ \{\{ \mathbf{a} \}, \{\mathbf{a},\mathbf{b} \}\}=\{\{\mathbf{c}\},\{\mathbf{c,d}\}\}$ es decir se tiene que:
\[[\{\mathbf{a}\} = \{\mathbf{c}\}\vee\{\mathbf{a}\} = \{\mathbf{c,d}\}]\wedge[\{\mathbf{a,b}\} = \{\mathbf{c}\}\vee\{\mathbf{a,b}\} = \{\mathbf{c,d}\}]\wedge[\{\mathbf{c,d}\} = \{\mathbf{a}\}\vee\{\mathbf{c,d}\} = \{\mathbf{a,b}\}]\]
Se tienen sólo las siguientes posibilidades:
	\begin{iteritem}
	\item \textbf{i)} $\{\mathbf{a}\}=\{\mathbf{c,d}\}$
	\item \textbf{ii)} $\{\mathbf{a,b}\}=\{\mathbf{c}\}$
	\item \textbf{iii)} $\{\mathbf{a}\}=\{\mathbf{c}\} \wedge \{\mathbf{a,b}\}=\{\mathbf{c,d}\}$ \\ 
	\end{iteritem}
En el caso \textbf{i)} \textbf{a=c=d}, además como $\{\mathbf{a,b}\}=\{\mathbf{c}\}\vee\{\mathbf{a,b}\}=\{\mathbf{c,d}\}$ necesariamente $\mathbf{b}=\mathbf{d}$. En el caso \textbf{ii)}, \textbf{a=b=c}, además como $\{\mathbf{c,d}\}=\{\mathbf{a}\} \vee \{\mathbf{c,d}\}=\{\mathbf{a,b}\}$ necesariamente \textbf{d=b}. En el caso \textbf{iii)} $\mathbf{a=c} \wedge [\mathbf{b}=\mathbf{c} \vee \mathbf{b}=\mathbf{d}]$. Si $\mathbf{b}=\mathbf{c}$, entonces $\mathbf{a=b=c}$, y así como:
	\[ \{\mathbf{c,d}\} = \{\mathbf{a}\} \vee \{\mathbf{c,d}\}=\{\mathbf{a,b}\} \]
	Se tiene necesariamente que $\mathbf{d}=\mathbf{b}$. En cualquiera de los casos \textbf{i),ii),iii)} se tiene que $\mathbf{a}=\mathbf{c} \wedge \mathbf{b}=\mathbf{d}$. 
\end{proof2}
\newpage
%%%%%%%%%%%%%%%%%%%%%%%%%%%%%%%%%%%%%%%%%%%%%%%%%%%%%%%%%%%%%%%%%%%%%%%%%%%%%%%%%%%%%%%%%%%%%%%%%%%%%%%%%%%%%%%%%%%%%%%%%%%%%%%%%%%%%%%%%%%%%%%%%%%%%%%%%%%%%%%%%%%%%%

\textbf{Definición:}\\
En general si $\mathcal{A,B}$ son conjuntos, $\mathbf{a}\in\mathcal{A}\wedge\mathbf{b}\in\mathcal{B}$, se define 
\textbf{la pareja ordenada cuyo primer elemento es a y cuyo segundo elementos es b} como $\{\{\mathbf{a},\{\mathbf{a,b}\}\}$. Tal
pareja se denota por $(\mathbf{a},\mathbf{b})$, y se tiene como antes que $(\mathbf{a,b})=(\mathbf{c,d})$, si y sólo si, 
$\mathbf{a}=\mathbf{c}\wedge\mathbf{c}=\mathbf{d}$.\\
}

Observe que $(\mathbf{a,b})\in\mathcal{P}(\mathcal{P}(\mathcal{A}\cup\mathcal{B}))$, ya que $\{\mathbf{a,b}\}\subset\mathcal{A}\cup\mathcal{B}$,
asímismo $\{\mathbf{a}\}\subset\mathcal{A}\cup\mathcal{B}$ $\therefore \{\mathbf{a,b}\}$ y $\{\mathbf{a}$ son elementos de 
$\mathcal{P}(\mathcal{A}\cup\mathcal{B})$. De ahí que $\{\{\mathbf{a}\}, \{\mathbf{a,b}\}\}\subset\mathcal{P}(\mathcal{A}\cup\mathcal{B})$, por 
lo cual $\{\{\mathbf{a}\},\{\mathbf{a,b}\}\}\in\mathcal{P}(\mathcal{P}(\mathcal{A}\cup\mathcal{B}))$.\\

\textbf{Definición:}\\
Por el axioma \textbf{ZF 6}, con la propiedad siguiente:
\[ \mathbf{P}(\mathbf{x}):\quad\exists\mathbf{a}\in\mathcal{A}\wedge\exists\mathbf{b}\in\mathca{B}\quad\mathbf{x}=(\mathbf{a,b})\]
\[ \exists\mathcal{D}\quad\forall\mathbf{x}[\mathbf{x}\in\mathca{D} \Longleftrightarrow\marhbf{x}\in\mathcal{P}(\mathcal{P}(\mathcal{A}\cup\mathcal{B}))\wedge\exists\mathbf{a}\in\mathcal{A},\exists\mathbf{b}\in\mathcal{B}\quad\mathbf{x}=(\mathbf{a,b})]\]
$\mathcal{D}$ se denomina el producto cartesiano de $\mathcal{A}$ y $\mathcal{B}$, y se denotará por $\mathcal{A}\times\mathcal{B}$.\\

\begin{proposition}
Sean $\mathcal{A,B}$ conjuntos. Se cumple que $\mathcal{A}\times\mathcal{B}=$\O, si y sólo si, $\mathcal{A}=$\O$\vee\mathcal{B}=$\O.
\end{proposition}
\begin{proof2}
	
	$\Longrightarrow)$ Por hipótesis $\mathcal{A}\times\mathcal{B}=$\O, se debe probar que $\mathcal{A}=$\O$\vee\mathcal{B}=$\O.\\
	Suponga que ello no ocurre, i.e. $\mathcal{A}\neq$\O$\wedge\mathcal{B}\neq$\O, entonces $\exists\mathbf{a}\in\mathcal{A}\vee\exists\mathbf{b}\in\mathcal{B}$,
	y así $(\mathbf{a,b})\in\mathcal{A}\times\mathcal{B}$. Con lo cual se tiene una contradicción.

	$\Longleftarrow)$ Ahora por hipótesis $\mathcal{A}=\O\vee\mathcal{B}=$\O, se debe probar que $\mathcal{A}\times\mathcal{B}=$\O. Suponga que 
	$\mathcal{A}\times\mathcal{B}\neq$\O. Entonces $\exists\mathbf{a}\in\mathcal{A},\exists\mathbf{b}\in\mathcal{B}$ 
	$(\mathbf{a,b})\in\mathcal{A}\times\mathcal{B}$ $\therefore \mathcal{A}\neq$\O$\wedge\mathcal{B}\neq$\O. Se tiene así una contradicción.

\end{proof2}

\textbf{Ejercicios:}
	\begin{iteritem}
	
	\item 1) Sean $\mathcal{A,B,C}$ conjuntos. Demuestre que si $\mathcal{C}\times\mathcal{D}\neq$\O, entonces 
	      $\mathcal{C}\times\mathcal{D}\subset\mathcal{A}\times\mathcal{B}$, si y sólo si, 
	      $\mathcal{C}\subset\mathcal{A}\wedge\mathcal{D}\subset\mathcal{B}$.
	
        \item 2) Sean $\mathcal{A,B,C}$ conjuntos, verifique lo siguiente:
		\begin{iteritem}
		\item a) $\mathcal{A}\times(\mathcal{A}\cup\mathcal{B})=(\mathcal{A}\times\mathcal{B})\cup(\mathcal{A}\times\mathcal{C})$.
		\item b) $\mathcal{A}\times(\mathcal{B}\cap\mathcal{C})=(\mathcal{A}\times\mathcal{B})\cap(\mathcal{A}\times\mathcal{C})$.
		\item c) $\mathcal{A}\times(\mathcal{B}$\textbackslash$\mathcal{C})=(\mathcal{A}\times\mathcal{B})$\textbackslash$(\mathcal{A}\times\mathcal{C})$.
		\item d) $\mathcal{A}\times\mathcal{C}=\mathcal{B}\times$, si y sólo si, $\mathcal{A}=\mathcal{B}$.
		\end{iteritem}
	\end{iteritem}
\newpage
%%%%%%%%%%%%%%%%%%%%%%%%%%%%%%%%%%%%%%%%%%%%%%%%%%%%%%%%%%%%%%%%%%%%%%%%%%%%%%%%%%%%%%%%%%%%%%%%%%%%%%%%%%%%%%%%%%%%%%%%%%%%%%%%%%%%%%%%%%%%%%%%%%%%%%%%

\textbf{Definición:}\\
Una \textbf{relación binaria} $\mathcal{R}$ es un conjunto cuyos elementos son parejas ordenadas.\\

Sea $(\mathbf{x,y})\in\mathcal{R}$, de donde $(\mathbf{x,y})=\{\{\mathbf{x}\},\{\mathbf{x,y}\}\}$, y $\{\mathbf{x,y}\}\in\cup\mathcal{R}$, ya que $\{\mathbf{x,y}\}\in(\mathbf{x,y})$ $\wedge(\mathbf{x,y})\in\mathcal{R}$. Así $\mathbf{x}\in\cup(\cup\mathcal{R})$, ya que $\mathbf{x}\in\{\mathbf{x,y}\}\wedge\{\mathbf{x,y}\}\in\cup\mathcal{R}$. Asímismo $\mathbf{y}\in\cup\cup\mathcal{R}$. Por el \textbf{Esquema Axiomático de Especificación ZF-6}:
\[\exists\mathcal{B}\quad\forall\mathbf{x}(\mathbf{x}\in\mathcal{B}\Longleftrightarrow\mathbf{x}\in\cup\cup\mathcal{R}\wedge\exists\mathbf{y}\in\cup\cup\mathcal{R}\quad(\mathbf{x,y})\in\mathcal{R})\]
$\mathcal{B}$ se denomina el \textbf{dominio de} $\mathcal{R}$ y se denotará por $Dom(\mathcal{R})$\\
Análogamente por el \textbf{Esquema Axiomático de Especificación ZF 6}: 
\[\exists\mathcal{D}\quad\forall\mathbf{x}(\mathbf{x}\in\mathcal{D}\Longleftrightarrow\mathbf{x}\in\cup\cup\mathcal{R}\wedge\exists\mathbf{t}\in\cap\cap\mathcal{R}\quad(\mathbf{t,x})\in\mathcal{R})\]
$\mathcal{D}$ se denomina el \textbf{rango de} $\mathcal{R}$ y se denota por $Ran(\mathcal{R})$.\\

\textbf{Definición:}\\
Una relación $f$ se dice que es una \textbf{función} si $\forall\mathbf{x}\in Dom(f)$ $\exists !\mathbf{y}\in Ran(f)$
tal que $(\mathbf{x,y})\in f$.\\
Tal que $\mathbf{y}$ se denomina \textbf{la imagen de x bajo f} y se denotará por $f(\mathbf{x})$.\\
\medskip
Dados dos conjuntos $\mathcal{A}$ y $\mathcal{B}$, se dice que $f$ es una \textbf{función de} $\mathcal{A}$
a $\mathcal{B}$, lo cual se denota por 
\begin{center}
$f:\mathcal{A}\longrightarrow\mathcal{B}$ si $Dom(f)=A$ y $Ran(f)\subset\mathcal{B}$.
\end{center}
\begin{iteritem}
\item $\mathcal{A}$ se dice \textbf{dominio} de $f$ y $\mathcal{B}$ se denomina el \textbf{codominio} de $f$. 
\item Se dice que $f$ es \textbf{suprayectiva} si: $Ran(f)=\mathcal{B}$.
\item Se dice que $f$ es \textbf{inyectiva} si: $\forall\mathbf{x,y}\in\mathcal{A}$, $\mathbf{x}\neq\mathbf{y}$ implica $f(\mathbf{x}) \neq f(\mathbf{y})$
\item Se dice que $f$ es una \textbf{biyección} si $f$ es \textbf{inyectiva} y \textbf{suprayectiva}.\\
Observe que $f$ es una \textbf{biyección}, si y sólo si, $\forall\mathbf{y}\in\mathcal{B}\quad\exists!\mathbf{x}\in\mathcal{A}\quad f(\mathbf{x})=\mathbf{y}$.\\
\end{iteritem}
\\
Sean $\mathcal{N}\subset\mathcal{A}$, $\mathcal{M}\subset\mathcal{B}$. Por el \textbf{Esquema Axiomático de Especificación ZF 6}:
\[\exists\mathcal{D}\quad\forall\mathbf{y}[\mathbf{y}\in\mathcal{D}\Longleftrightarrow\mathbf{y}\in\mathcal{B}\wedge\exists\mathbf{x}\in\mathcal{N}\quad\mathbf{y}=f(\mathbf{x})]\]
\begin{iteritem}
\item $\mathcal{D}$ se denota por $f[\mathcal{N}]$ y se denomina \textbf{La imagen de} $\mathcal{N}$ \textbf{bajo} $f$.
\item Por el mismo \textbf{ZF 6} 
	\[\exists\mathcal{E}\quad\forall\mathbf{x}[\mathbf{x}\in\mathcal{E}\Longleftrightarrow\mathbf{x}\in\mathcal{A}\wedge\exists\mathbf{y}\in\mathcal{M}\quad f(\mathbf{x})=\mathbf{y}]\]
$\mathcal{E}$ se denota por $f^{-1}[\mathcal{M}]$ y se denomina \textbf{La imagen inversa de} $\mathcal{M}$ \textbf{bajo} $f$. 
\end{iteritem}
\newpage

%%%%%%%%%%%%%%%%%%%%%%%%%%%%%%%%%%%%%%%%%%%%%%%%%%%%%%%%%%%%%%%%%%%%%%%%%%%%%%%%%%%%%%%%%%%%%%%%%%%%%%%%%%%%%%%%%%%%%%%%%%%%%%%%%%%%%%%%%%%%%%%%%%%%%%%%%%%%%%%%%%

\begin{proposition}
	Considere una función $f:\mathcal{A}\longrightarrow\mathcal{B}$. Sean $\mathcal{N,M}$ subconjuntos de $\mathcal{A,B}$ respectivamente.
	Se cumplen las siguientes condiciones:
	\begin{iteritem}
	\item i) $\mathcal{N}\subset f^{-1}[f[\mathcal{N}]]$. Si $f$ es inyectiva, entoces $\mathcal{N}=f^{-1}[f[\mathcal{N}]]$.
	\item ii) $f[f^{-1}[\mathcal{M}]]\subset\mathcal{M}$. Si $f$ es suprayectiva, entonces $f[f^{-1}[\mathcal{M}]]=\mathcal{M}$.
	\end{iteritem}
\end{proposition}
\begin{proof2}
	i) Sea $\mathbf{x}\in\mathcal{N}$, entonces $f(\mathbf{x})\in f[\mathcal{N}]$ $\therefore$ $\mathbf{x}\in f^{-1}[f[\mathcal{N}]]$.
	Por lo tanto $\mathcal{N}\subset f^{-1}[f[\mathcal{N}]]$.\\
\end{proof2} 

























































































































\begin{proposition}
    Sea $(N,R)$ un conjunto parcialmente ordenado tal que toda cadena
    contenida en $N$ tiene un supremo en $N$. Sea $f:N\rightarrow N$ una
    función tal que $\forall x \in N \quad x \prec f(x)$. Entonces, existe
    algún $n \in N \quad f(n)=n$.
\end{proposition}
\begin{definition}
    Dado $u \in N$ se dice que $A \subset N$ es \textbf{admisible} con relación a $u$
    si se cumple: 
	\begin{enumerate}
        \item $u \in A$.
        \item $f(A) \subset A$.
        \item Si $B$ es una cadena contenida en $A$, entonces en $A$ mismo
            existe el supremo de $B$.
    \end{enumerate}
\end{definition}
\begin{proof}
    Dado $u \in N$, sea $M$ el conjunto de todos los conjuntos admisibles de
    $u$ en relación a $N$. \\
    Definamos: $S= \bigcap\limits_{x\in M} x$.  Veamos que:\\
    \begin{enumerate}
        \item $u \in S$ ya que $u \in S $
        \item Sea $t \in S$, $t \in  x$, $\,\therefore\,$ $f(t) \in x \,
            \forall x  \in M$, lo cual muestra que $f(S) \subset S$.
        \item Sea $B$ una cadena contenida en $S$. $B \subset x \, \forall x
            \in N$ $\, \therefore \,$ $\sup B \in S$.
    \end{enumerate}
    Así $S$ es admisible con respecto a $u$. Observe que si $S' \subset S \,
    \land \, S'$ es $u$-admisible, entonces $S' =S$. \\
    \textbf{Hecho 1} $u$ es el primer elemento de $S$, es decir, $u\prec x
    \, \forall x \in S$.
    Sea $A= \{ x \in S\, | \, u \prec x\}$. Se demostrará que $A =S$.
    Como $A \subset S$, bastará probar que $A$ es $u$ admisible.
    \\
    \begin{enumerate}
        \item Como $R$ es en particular reflexiva: $u \prec u$. Además $u
            \in S$. Así $u \in A$.
        \item Sea $x \in A$, veamos que $f(x) \in A$ ya que $u \prec x$, por
            transitividad $u \prec f(x)$. Así $x \in S$ y al ser
            $u$-admisible, $f(x) \in A$. Así $f(A) \subset A$.
        \item Sea $B$ una cadena contenida en $A$, 
    \end{enumerate}
\end{proof}


\section{Problemas}
\begin{definition}
    Sean $R_1$, $R_2$ relaciones en $A$, se define la relación
    \textbf{composición} $R_1 \circ R_2$
    \[
        R_1 \circ R_2 = \{\, (a,b) \in A \times A  | (a,c) \in R_1 \land (c,b)
        \in R_2 , \, c \in A\,\}
    \]
\end{definition}
\begin{definition}
    Sea $R$ una relación binaria en $A$, se define la relación
    \textbf{inversa} $R^{-1}$ como sigue:
    \[
        (a,b) \in R^{-1} \Leftrightarrow (b,a) \in R
    \]
\end{definition}
\begin{problem}
    Sean $R$, $S$ y $T$ relaciones binarias en $A$, demuestre que $(R \circ
    S) \circ T = R \circ (S \circ T)$. Muestre que si $R$ y $S$ son
    relaciones de
    equivalencia, $R \circ S$ no es
    necesariamente una relación de equivalencia. 
\end{problem}
\begin{sol}
    Considere $(a,b) \in (R \circ S) \circ T$ arbitrario. Existe entonces
    $c\in A$ tal que
    $(a,c) \in (R \circ S)$, $(c,b) \in T$. Existe entonces $d\in A$ tal que
    $(a,d) \in R$, $(d,c) \in S$. Así $(d,b) \in (S \circ T)$, y como $(a,d)
    \in R$, $(a,b) \in R \circ (S \circ T)$. $(R \circ
    S) \circ T \subset R \circ (S \circ T)$; la segunda contención se
    demuestra analogamente.
    \hfill$\square$ \\
    Considere $A=\{\clubsuit, \spadesuit, \blacklozenge\}$, 
    $R=\{(\clubsuit, \clubsuit), (\spadesuit,\spadesuit),
    (\blacklozenge,\blacklozenge),(\clubsuit,\spadesuit),(\spadesuit,\clubsuit)\}$
    y 
    $S=\{(\clubsuit, \clubsuit), (\spadesuit,\spadesuit),
        (\blacklozenge,\blacklozenge),(\blacklozenge, \clubsuit), (\clubsuit,
    \blacklozenge)\}$.\\
    Veamos que $(\spadesuit, \blacklozenge) \in R \circ S$ pero $(\blacklozenge,
    \spadesuit) \notin R \circ S$. Lo cual asegura que $R \circ S$ no es de
    equivalencia.
\end{sol}
\begin{problem}
    Sea $R$ una relación binaria en $A$, $R$ es de equivalencia sí y solo sí $R=R\circ R \, \land \, R=R^{-1}$.
\end{problem}

\begin{sol}
    $\Rightarrow$ \\
    Se probará que $R=R\circ R$ y $R=R^{-1}$. \\
    Suponga que $R$ es una relación de equivalencia, y suponga que $x=(a,b) \in
    R$, veamos que $(b,b) \in R$ al ser $R$ reflexiva, entonces $(a,b) \in R
    \circ R$ entonces $R \subset R \circ R$, como $R$ es simétrica $(b,a) \in
    R$, así $(a,b) \in R^{-1}$ $\therefore$ \, $R \subset R^{-1}$. Ahora suponga que
    $x = (a,b) \in R\circ R$, entonces $\exists c \in A \, (a,c) \in R \land
    (c,b) \in R$, pero al ser $R$ transitiva $(a,b) \in R$ $\therefore$ \, $ R\circ R
    \subset R$. Finalmente suponga que $x=(a,b) \in R^{-1}$, entonces $(b,a) \in
    R$, pero como $R$ es simétrica, $(a,b)\in R$ así $R^{-1} \subset R$, y $R=R
    \circ R \, \land \, R = R^{-1}$. \\
    $\Leftarrow$ \\
    Suponga ahora que $R=R\circ R$ y $R=R^{-1}$. Se probará que $R$ es de
    equivalencia. \\
    Suponga que $(a,b) \in R$, arbitrario; como $R=R^{-1}$, $(b,a)\in R$ por lo tanto es
    simétrica. Veamos que al ser $R=R\circ R$ $(a,a) \in R \, \land \, (b,b)
    \in R$, y por ser $(a,b)$ arbitrario, $R$ es reflexiva. Finalmente suponga
    que $(a,b) \in R \, \land \, (b,c) \in R$, entonces $(a,c) \in R \circ R$,
    $(a,c)\in R$, así $R$ es transitiva, por lo tanto, de equivalencia.
\end{sol}
\begin{problem}
    Sea $R$ una relación binaria reflexiva y transitiva en $A$. Demuestre
    que $S = R \cap R^{-1}$ es de equivalencia.
\end{problem}
\begin{sol}
    Considere $a\in A$ arbitrario, como $R$ es reflexiva, $(a,a) \in R$,
    $(a,a) \in R^{-1}$ $\,\therefore \,$ $(a,a) \in S$. Suponga que
    $(a,b)\in S$, entonces $(a,b)\in R$, $(a,b) \in R^{-1}$, así $(b,a) \in
    R^{-1} \, \land \, (b,a) \in R$ $\, \therefore \,$ $(b,a) \in S$.
    Finalmente suponga que $(a,b)\in S, \, (b,c) \in S$ como $R$ es
    transitiva, $(a,c)\in R$, como $S$ es simétrica $(c,b), (b,a)\in R$ así,
    $(c,a) \in R$ por lo tanto $(a,c) \in R^{-1}$, $(a,c)\in S$. Así, $S$ es
    una relación de equivalencia.
\end{sol}
\begin{problem}
    Sean $R$ y $S$ relaciones en $A$ y $B$ respectivamente. Se define la
    relación $R \times S$ como el conjunto $\{ \, ((a,b),(c,d)) \in (A\times
    B) \times (A \times B) \, | \, (a,c) \in R \, , \, (b,d) \in S \,\}$. Si
    $R$ y $S$ son de equivalencia, demuestre que $R \times S$ es de
    equivalencia.
\end{problem}
\begin{sol}
    Considere $(a,b)$ arbitrario en $A \times B$, como $R$,$S$, son de
    equivalencia $(a,a) \in R$, $(b,b)\in S$; así $((a,b),(a,b))\in R \times
    S$, $R \times S$ es reflexiva. Considere ahora a $((a,c),(b,d))\in
    R\times S$ arbitrario. Por la definición $(a,b) \in R \, \land \, (c,d)
    \in S$; así $(b,a) \in R$, $(d,c) \in S$, de forma que $((b,d),(a,c))
    \in R \times S$, $R \times S$ es simétrica. Finalmente considere
    $((a,c),(b,d)$, $((b,d), (f,e)) \in R \times S$. Las relaciones
    $(a,b)$, $(b,f) \in R$, así $(a,f) \in R$; y $(c,d)$, $(d,e) \in S$ con
    $(c,e) \in S$. Ésto último nos asegura que $((a,c),(f,e))\in R \times
    S$, es decir, $R \times S$ es transitiva; lo que demuestra que es de
    equivalencia.
\end{sol}
\begin{problem}
    Sea $A$ un conjunto, se define la diagonal de $A$, $\Delta A = \{(a,a)
    \in A \times A | a \in A\}$. Sea $R$ una relación binaria, demuestre que
    $R$ es un preorden sí y solo sí $\Delta A \subset R \, \land \, R =
    R\circ R $.
\end{problem}
\begin{sol}
    $\Rightarrow$ \\
    Suponga que $R$ es un preorden, como $R$ es reflexivo $(a,a)\in R
    \forall a \in A$ así $\Delta A \subset R$. Considere $(a,b) \in R$, como
    $R$ es reflexivo, $(b,b) \in R$, por lo tanto $(a,b) \in R \circ R$. \\
    Ahora considere $(a,b) \in R \circ R$, entonces existe $c \in A$ tal que 
    $(a,c)\in R \land (c,b) \in R$; pero al ser $R$ transitiva $(a,b) \in
    R$, así $R=R\circ R$.\\
    $\Leftarrow$ \\
    Suponga ahora que $\Delta A \subset R \land R = R \circ R$. Trivialmente
    $R$ es reflexiva ya que $(a,a) \in \Delta A \forall a \in A$. Considere
    ahora $(a,b)$, $(b,c) \in R$ por la definición $(a,c) \in R \circ R = R$
    $\therefore$ $R$ es transitiva.
\end{sol}
\begin{remark}
    El siguiente problema no tiene mucho caso según como se planteó en la
    lista, la idea interesante del problema sucede cuando se retira al 1, y
    así lo voy a resolver.
\end{remark}
\begin{problem}
    Defínase $R$ en $A=\mathbb{Z}^{+}\setminus\{1\}$ como $\{(n,m) \in R \, | \, m | n\}$.
    Muestre que $R$ es un orden parcial, que toda cadena tiene cota superior
    y determine el conjunto de elementos maximales. 
\end{problem}
\begin{sol}
    Sea $n \in A$, trivialmente $n | n$, así $(n,n) \in R$. Considere
    ahora $(m,n)$, $(n,p) \in R$, entonces $m=na$, $n=pb$$\Rightarrow$
    $m=pab$, $p | m$, así $(m,p) \in R$. Finalmente suponga que $(m,n) \in R
    \land (n,m) \in R$, entonces $m | n \,\land \, n | m$; lo cual implica
    $n \geq m \, \land \, n \leq m$. así $n=m$. Lo cual prueba que $R$ es un
    orden parcial. 
    Sea $B$ una cadena no vacía de $A$, veamos que $B \subset \mathbb{N}$,
    así $B$ tiene un primer elemento $u$ respecto al orden de los naturales.
    Suponga que existe $v \in B$ tal que $(u,v)\in R$, entonces $v | u$ y $v
    \leq u$, como $u$ era un primer elemento $v=u$. Así, en particular, $u$
    es cota superior de $B$. Como $B$ fue arbitrario, cualquier cadena está
    acotada superiormente. Ahora veamos que el conjunto de elementos
    maximales de $A$ es el conjunto $P=\{p \, | \,$ $p$ es primo $\}$. Sea
    $p \in P$, entonces si $(p,b) \in A$, $b | p$; por la definición de
    primo, $b=p \, \lor \, b=1$. Como $b \in A$, $b =p$, lo cual prueba que
    $p$ es un elemento maximal de $A$. Se puede proceder de la definición de
    primo, recíprocamente, para demostrar que si un elemento es maximal,
    entonces es primo.
    %Por contradicción, suponga que existe una cadena $C$ que
    %carece de cota superior, es decir, $\nexists a \in A , b\prec a \quad \forall
    %b \in C$. En particular, $\forall b \in C \quad \exists c \neq b \in C$ tal que $b
    %\prec c$. Tome un $b_0$ arbitrario pero fijo en $C$, y considere la
    %subcadena $\{c_1, c_2, \ldots, c_{b_0}\}$ tal que $b_0 \prec c_1 \prec
    %\ldots \prec c_{b_0}$ y $c_i \neq c_j$, $i \neq j$. Así como cada $c_i
    %|c_{i-1}$, $c_i < c_{i-1}$, así se construye una sucesión decrecientes de $b_0$ 
    %números positivos menores a $b_0$ lo cual contradice que $C\subset
    %\mathbb{Z}^{+}$, por lo tanto todas las cadenas tienen una cota superior. \\
    %Es fácil ver que el conjunto de elementos maximales en $\mathbb{Z}^{+}$ es el
    %conjunto $\{1\}$ ya que $\forall x \in \mathbb{Z}^{+} \quad 1 | x$.
\end{sol}
\begin{problem}
    Suponga que $R$ es un buen orden en $A$, demuestre que si $A$ no es
    finito $R^{-1}$ no es un buen orden.
\end{problem}
\begin{sol}
    Suponga que $A$ no es finito, entonces para cualquier conjunto finito
    $B_i$ no vacío, existe un conjunto finito $B_{i+1}$ que lo contiene
    propiamente $\forall i \in \mathbb{N}$. $A$ es no vacío, así podemos
    tomar un $x_0$ cualquiera y tomar a $B_0= \{x_0\}$, y a cada
    $B_{i+1}=B_i \cup \{x_i\} \quad x_i\notin B_i$. Veamos que cada $B_i$ es
    una cadena en $A$, y consideremos a $C=\bigcup\limits_{x \in B_i} x
    \quad i \in \mathbb{N}$. Como $C$ es una cadena, podemos ordenarlos en
    una sucesión $\{x_k\}$ tal que $(x_k,x_{k+1}) \in R\, \land \, x_k \neq
    x_{k+1} \forall k \in \mathbb{N}$. Veamos que $(x_{k+1}, x_k)\in R^{-1} \forall k
    \in \mathbb{N}$. Así por el problema \ref{pr:pr11}, $R^{-1}$ no puede
    ser un buen orden.
\end{sol}
\begin{problem}
    Considere a $A$ el conjunto de las sucesiones en $\mathbb{R}$, definimos el
    orden lexicográfico como $R=\{(\{x_k\},\{y_k\}) \, | \, x_k = y_k
    \forall k \, \lor \, x_n < y_n$ si $x_i = y_i \forall i < n  \}$.
    Demuestre que $R$ es un orden total en $A$.
\end{problem}
\begin{sol}
    Considere $\{x_k\}$ arbitrario en $A$. $(\{x_k\},\{x_k\})\in R$. Así,
    $R$ es reflexiva. Considere ahora $(\{x_k\},\{y_k\})\in
    R$, $(\{y_k\},\{z_k\})\in R$.
    Podemos considerar entonces los casos siguientes:
    \begin{itemize}
        \item  $x_k = y_k\forall k \, \land \, y_k = z_k \, \forall k$ \\
            Trivialmente $\{x_k\}=\{z_k\}$.
        \item  $x_k = y_k\forall k \, \land \, y_i 
            = z_i \, \forall i < n $ si $x_n<y_n$\\
            Como $\{x_k\}=\{y_k\}$. $x_i < z_i \, \forall \, i < n$ si
            $x_n<z_k$.
        \item $x_i = y_i \forall i<n$ si $x_n < y_n \, \land \, y_k = z_k \,
            \forall k$. \\
            Como $\{y_k\} = \{z_k\}$, se tiene $x_i = z_i \forall i<n$ si
            $x_n < z_n$.
        \item $x_k = y_k \forall k<n$ si $x_n < y_n \, \land \, y_i = z_i
            \forall i<n$ si $y_n < z_n$ \\
            Como el orden en $\mathbb{R}$ es ya un orden total, por
            transitividad. $x_k=z_k \forall k<n$ si $x_n<z_n$.
    \end{itemize}
    En cualquiera de los casos concluímos que $(\{x_k\},\{z_k\})\in R$. Así
    $R$ es transitiva.
    Suponga ahora que $(\{x_k\},\{y_k\}) \in R$ y que $(\{y_k\},\{x_k\}) \in
    R$. 
    En el primer caso de la disyunción $\{x_k\} = \{y_l\}$, no hay nada que
    hacer. En el segundo caso, $x_k=y_k \forall k<n$ si $x_n<y_n$, 
    $y_k=x_k \forall k<n$ si $y_n<x_n$. En tal caso no existe $n$ cual $x_n
    \neq y_n$, entonces $x_n =y_n \, \forall n $. Así $R$ es un orden parcial.
    Demostremos ahora que es una cadena. \\
    Sean $x=\{x_k\}$, $y=\{y_k\}$ sucesiones arbitrarias, en el caso que sean
    iguales, trivialmente $(x,y)\in R$. Si son distintas, existe un primer
    $k_0$ (dado que cualquier subconjunto de los naturales tiene un primer
    elemento) tal que $x_{k_0}\neq y_{k_0} \, \land \, x_i = y_i \quad
    \forall i < k_0$. En dado caso $(x,y)\in R$ ó $(y,x)\in R$.


\end{sol}
\begin{problem}
    Sea $A$ un conjunto finito, demuestre que todo orden total es un buen
    orden.
\end{problem}
\begin{sol}
    Sea $R$ un orden total en $A$, y $B$ una cadena arbitraria no vacía. Al
    ser $A$ finito, $B$ es finito, se demostrará que $B$ tiene primer
    elemento. Suponga que $B$ tiene $n$ elementos y que no tiene un primero, es decir, para cualquier $a \in B \quad
    \exists b, c \neq a \in B \quad , a \prec b \land c \prec a$. Considere
    un $c_0 \in B$ arbitrario, se puede entonces construir una secuencia
    $\{c_0, c_1, \ldots, c_n \}$ de forma inductiva ya que $B$ es no vacío y
    $B$ no tiene un primer elemento, con la condición de que cada $c_i \prec
    c_{i-1}$, $c_i \neq c_{i-1}$ $i=\{0,1,\ldots,n\}$. Como observación ésta
    secuencia es un subconjunto de $B$. Veamos que por transitividad cada
    $c_i \prec c_j$ si $j>i$. De igual forma, todos son distintos, pero esto
    contradiría que $B$ tiene $n$ elementos ya que la secuencia tiene $n+1$.
    Así $B$ tiene primer elemento. Como fue arbitrario $R$ es un buen orden.
\end{sol}
\begin{problem}
    Sea $(A,R)$ un conjunto bien ordenado. Demuestre que no existe una
    sucesión $\{x_n\}$ tal que $x_{n+1} \prec x_n$ $\, \land \, x_{n+1} \neq
    x_n \quad \forall n \in
    \mathbb{N}$
    \label{pr:pr11}
\end{problem}
\begin{sol}
    Procediendo por contradicción, suponga que dicha sucesión $\{x_k\}$
    existe. Por la forma en que está definida, el conjunto $B =\{x_k \forall
    k \in \mathbb{N}\}$ es una cadena en $A$. Así $B$ tiene un primer
    elemento. Existe entonces $i_0 \in \mathbb{N}$ tal que $x_k \prec
    x_{i_0} \forall k \in \mathbb{N}$, se llega aquí a una contradicción
    ya que $x_{i_0+1} \prec x_{i_0} \, \land \, x_{i_0 +1} \neq x_{i_0}$. 
    Así $R$ no sería un buen orden, por lo tanto dicha sucesión no puede
    existir.
\end{sol}
