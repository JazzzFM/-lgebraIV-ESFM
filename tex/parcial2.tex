    \section{Los Números Naturales}
    \begin{definition}
        Definimos la función sucesor de un conjunto $x$ como $x^+ = x \cup
        \{x\}$.
        \\
        Se denota al conjunto $\emptyset$ como 0, y al conjunto $\emptyset^+$ como 1.
    \end{definition}
    \begin{axiom}{de Infinitud}
        \[
            \exists A \quad x \in A \, \Leftrightarrow \, \emptyset \in x \, \land
            \, x \in x^+
        \]
    \end{axiom}
        \begin{definition}
            Se dice que un conjunto $X$ es un conjunto sucesor si
            \begin{enumerate}
                \item $\emptyset \in X$
                \item $ \forall x \, (x \in X \, \Rightarrow \, x^{+} \in X)$
            \end{enumerate}
    \end{definition}
    Así el axioma del infinito postula la existencia de un conjunto sucesor. \\
    Sea $A$ un conjunto sucesor.
    Note que la existencia de $A$ está garantizada por el axioma del infinito.
    \\
    Sea $ B = \{ \, x \in \mathcal{P}(A) \, | \, \emptyset \in x \, \land \quad
    [\forall z \, (z \in x \, \Rightarrow \, z^+ \in x)]\}$. \\
    Por el axioma de especificación el conjunto $B$ es un conjunto. Se definirá
    el conjunto $\mathbb{N}$ de los números naturales como $\bigcap B$.
    \begin{proposition}
        La definición de $\mathcal{N}$ es independiente de la elección de $A$.
    \end{proposition}
    \begin{proof}
        Observe que 
        \begin{enumerate}
            \item $\emptyset \in x \quad \forall x \in B$. $\therefore \quad \emptyset
                \in \mathbb{N}$.
            \item Si $x \in \mathbb{N}$, entonces $x \in z \, \forall z \in B$.
        \end{enumerate}
        Como todo elemento de $B$ es un conjunto sucesor, se tiene que $x^+ \in
        z \, \forall z \in B \quad \therefore \, x \in \mathbb{N}$. Así
        $\mathbb{N}$ es un conjunto sucesor.  \\
        Sea $H$ un conjunto sucesor. Veamos que $A\cap H$ es un conjunto
        sucesor. Además:
        \begin{align*}
            A \cap H &\in \mathcal{P}(A) \\
            \therefore A \cap H &\in B \\
        \end{align*}
        Lo anterior implica que $\mathbb{N} \subset A \cap H \subset H$.
        Así $\mathbb{N} \subset X \quad \forall X $, $X$ conjunto sucesor.
    \end{proof}
    Se tiene:
    \begin{enumerate}
        \item $ 0 \in \mathbb{N}$
        \item $ \forall x \quad (x \in \mathbb{N} \, \rightarrow \,x^+ \in
            \mathbb{N} $
        \item Si $S \subset \mathbb{N}$ que satisface:
            \begin{enumerate}
                \item $ 0 \in S$
                \item $ \forall x \quad (x \in S \, \rightarrow \,x^+ \in S$
            \end{enumerate}
        \item $\forall x \quad ( x \in \mathbb{N} \rightarrow x^+ \neq 0 )$ ya
            que $ x \in x^+$
        \item $\forall x \, \forall y \quad [(x \in \mathbb{N} \, \land \, y \in
            \mathbb{N} \, \land \, x^+ = y^+ ) \Rightarrow \, x = y]$
    \end{enumerate}
    La última afirmación se verificará a partir de la siguiente definición y los
    siguientes dos lemas:
    \begin{definition}
        Un conjunto $A$ se dice \textbf{transitivo} sí $\forall x \, \forall y [
        (x \in A \, \land \, y \in x ) \, \, \Rightarrow y \in A]$.
    \end{definition}
    \begin{remark}
        Son equivalentes las siguientes afirmaciones:
        \begin{enumerate}
            \item $A$ es un conjunto transitivo.
            \item $\forall  x \, (x \in A \, \Rightarrow x \subset A)$.
            \item $ \bigcup A \subset A $.
            \item $ A \subset \mathcal{P}(A)$.
        \end{enumerate}
    \end{remark}
    \begin{proof}
        \textit{i $\rightarrow$ ii} \\
        Si $x \in A \, \land y \in x$, al ser $A$ transitivo, $y \in A \,
        \therefore \, x \subset A$.
        \textit{ii $\rightarrow$ iii} \\
        \textit{iii $\rightarrow$ iv} \\
        \textit{iv $\rightarrow$ i} \\
    \end{proof}
    \begin{definition}
        Por número natural se entiende cualquier $x$, $x \in \mathbb{N}$.
    \end{definition}
    \begin{lemma}\label{lem:p1l1}
        Todo elemento de $\mathbb{N}$ es un conjunto transitivo.
    \end{lemma}
    \begin{proof}
        Sea 
        \[
            S = \{n \in N \, | \, \forall x \, (x \in n \, \Rightarrow \, x
            \subset n)\}
        \]
        Observe que:
        \begin{enumerate}
            \item $0 \in S$ ya que $0 \in \mathbb{N}$ y $\forall x \, ( x \in \emptyset
                \, \Rightarrow \, x \subset \emptyset )$
            \item Suponga que $n \in S$, entonces $n \in \mathbb{N}$. Además
                $n^+ \in \mathbb{N}$. Sea $x \in n^+$ entonces $x \in n \, \lor
                \, x = n$. 
                \begin{itemize}
                    \item Si $x \in n$ entonces, dado que $n \in S$, $x \subset
                        n$. Además, $n \subset n^+$, por lo cual $x \subset
                        n^+$.
                    \item Si $x=n$, entonces $x \subset n^+$
                \end{itemize}
        \end{enumerate}
        Así $S = \mathbb{N}$ que era lo que se quería demostrar.

    \end{proof}
    \begin{lemma}
        Ningún número natural es subconjunto de alguno de sus elementos.
        \[
            \forall n \, [(n \in \mathbb{N} \, \land \, x \in n) \Rightarrow \,
            n  \setminus x \neq \emptyset]
        \]
    \end{lemma}
    \begin{proof}
        Sea $S = \{ n \in \mathbb{N} \, | \, \forall x \, (x \in n \Rightarrow n
     \setminus x \neq \emptyset\}$
    \begin{enumerate}
        \item $ 0 \in S$ ya que $\emptyset \in \mathbb{N} \, \land \, \forall x
            (x \in \emptyset \Rightarrow \emptyset  \setminus x \neq \emptyset )$
        \item Suponga que $n \in S$ , entonces, en particular $n \in \mathbb{N}
            \, \therefore n^+ \in \mathbb{N}$. Se verificará que $ n^+ \in S$.
            Sea $x \in n^+$, entonces $x \in n \, \lor \, x = n$. 
            \begin{itemize}
                \item Suponga que $x \in n$. Entonces dado que $n \in S$, $n  \setminus x
                    \neq \emptyset$. En consecuencia $n^+  \setminus x \neq \emptyset$ ya
                que si $n^+  \setminus x = \emptyset$ todo elemento de $n^+$ pertenecería
                a $x$, en particular, como $n \subset n^+ $ todo elemento de $n$
                pertenecería a $x$ lo cual contradice que $ n  \setminus x \neq
                \emptyset$. 
            \item Suponga ahora que $x=n$. Así, procediendo por reducción al
                absurdo, considere $n^+ \subset n$. Como $ n \in S \, \land \, n
                 \setminus n = \emptyset$ se tiene $n \notin n$. Como $n \in n^+$, se
                tiene
                $ \neg(n^+ \subset n)$, así como $n^+ \subset n$, se tiene una
                contradicción $\therefore$ $n^+  \setminus x \neq \emptyset$.
        \end{itemize}
    \end{enumerate}
    Así $S = \mathbb{N}$
    \end{proof}
    Los lemas anteriores permiten justificar la propiedad \textit{e}.
    \begin{proof}
        Sean $x$, $y$ números naturales tales que $x^+ = y^+$.  Así por ser
        elementos de $\mathbb{N}$ se tienen dos posibilidades $x=y$ ó $x \in y
        \, \land \, y \in x$ lo cual implica del lema \ref{lem:p2l1} que $x=y$.
    \end{proof}
    Las propiedades 1, 2, 3, 4, y 5 fueron postuladas en 1889 por Peano como un
    conjunto de axiomas, y la existencia de un conjunto con dichas propiedades.
    Consigo se construyeron los conjuntos
    $\mathbb{Z},\mathbb{Q},\mathbb{R},\mathbb{C}$ y sus propiedades analíticas y
    aritméticas. Peano reconoce que sus axiomas fueron postulados un año antes
    por Richard Dedekind.
    \vspace{1cm}
    \hline
    \vspace{1cm}
    
    Considere la función: $ S : \mathbb{N} \rightarrow \mathbb{N}$. $S(n)=n^+$.
    \\
    \begin{itemize}
        \item $S$ es inyectiva, ya que si $S(n)= S(m)$, de la propiedad
            5 se sigue que $n=m$.
        \item $S$ no es suprayectiva, ya que la propiedad 4 indica $0 \notin
            S(\mathbb{N})$. De hecho $S(\mathbb{N}) = \mathbb{N} \setminus \{0\}$
            Veamos que, como $0\notin S(\mathbb{N})$ se tiene que $S(\mathbb{N})
            \subset \mathbb{N}$. \\
            Además siendo $M=S(\mathbb{N})\bigcup\{0\}$ se tiene que:
            \begin{enumerate}
                \item $0 \in M$
                \item si $n \in M$ entonces $n^+ \in M$.
            \end{enumerate}
    \end{itemize}
    
    La propiedad 3 implica $\mathbb{N}=M$ $\therefore$ si $x \in N\setminus \{0\}$, entonces
    $x \in  M\setminus \setzero$ por lo cual $x \in S(\mathbb{N})$. $\therefore
    \, S(\mathbb{N}) = \mathbb{N} \setminus \setzero$\\


    \begin{definition}
        Sea $X$ un conjunto; $g:\, X \rightarrow X$ una función, y $x_0 \in X$.
        Decimos que la terna $(X, g, x_0)$ es un \textit{sistema de Peano} si:
        \begin{enumerate}
            \item $g$ es inyectiva
            \item $g(X) = X \setminus \{x_0\}$
            \item $[\, A\subset X \, \land \, x_0 \in A \, \land \, ( \, g(x)
                \in A \, \, \forall x \in A \, ) \, ] \Rightarrow A=X$
        \end{enumerate}
    \end{definition}
    \begin{remark}
        $(\mathbb{N}, S, 0)$ es un sistema de Peano.
    \end{remark}
    \begin{definition}
        Dado un conjunto $A$, una \textit{operación binaria en} $A$ es una
        función $f: \, A\times A \rightarrow A$.
        \begin{itemize}
            \item Si $B \subset A$, se dice que $B$ es \textit{cerrado bajo} $f$
                si $f(B\times B) \subset B$.
            \item $f$ es \textit{conmutativa} sí $f(x,y)=f(y,x)\quad \forall \,
                x, y \in A$
            \item $f$ es \textit{asociativa} sí $f(x, f(y,z)) = f(f(x, y), z)
                \quad \forall \, x, y, z \in A$
        \end{itemize}
    \end{definition}
    \begin{proposition}{(de recurrencia)}\label{pr:recurrencia} Sea $A$ un conjunto no vacío, $G: \, A
        \rightarrow A$ una función y $a \in A$. Entonces existe una única
        función $F: \, \mathbb{N} \rightarrow A$ tal que $\forall \, n \in
        \mathbb{N} \quad F(n^+)= G(F(n))$ y $F(0) = a$.
    \end{proposition}
    \begin{proof}
        Sea $C = \{ \, T \in  \mathcal{P}(\mathbb{N} \times A) \, | \, (0,a) \in
        T \, \land \, [\,(n,b) \in T \rightarrow (n^+, G(b))\in T \,]\, \}$
        Veamos que $C$ es no vacío ya que $\mathbb{N}\times A \in C$. \\
        Sea $F = \bigcap C$. Veamos que $F \in C$ ya que $(0,a) \in T \, \forall \,
        T \in C$ y si $(n,b) \in F$, se tiene que $(n, b) \in T \, \forall \,T
        \in C \, \therefore \, (n^+, G(b)) \in T \, \forall \, T \in C \,
        \therefore \, (n^+, G(b))\in F$ \\
        Se verificará que $F$ es función. \\
        Sea $M= \{\, n\in \mathbb{N} \, | \, \exists ! \, b\in S \quad (n,b) \in F
        \, \}$ \\
        \textbf{Hecho 1:} $0 \in M$ \\
        Veamos que $0 \in \mathbb{N}$, y $(0,a) \in F$. \\
        Suponga que existe $b \in A$ tal que  $(0,b) \in F \, \land \, b \neq
        a$. Sea $F_b = F \setminus \{(0,b)\}$. Se tiene que $F_b \neq F$, ya que
        $(0,b) \in F\setminus F_b$; además $F_b \subset F$. \\
        Como $F \subset \mathbb{N} \times A$ se tiene $F_b \subset \mathbb{N}
        \times A$; además $(0,a) \in F_b$. \\
        Sea $(n,b) \in F_b$, enconces $(n,b) \in F \, \therefore$ como $F \in C
        \, \Rightarrow \, (n^+, G(n))\in F$ Además como $n^+ \neq 0$, se tiene
        $(n^+, G(b))\in F_b\, \therefore \, (n^+, G(b))\in F \setminus
        \{(0,b)\}$ i.e $(n^+, G(b))\in F_b$ \\
        Se ha probado que $F_b \in C \, \therefore \, F\subset F_b$. Así,
        $F=F_b$ lo cual es una contradicción, por lo tanto $a=b$, por ende $0
        \in M$. \\
        \textbf{Hecho 2:} $n^+ \in M \quad \forall n \in M$ \\
        Sea $n \in M$, entonces $\exists ! b\in A$ tal que $(n,b) \in F$. Como
        $F \in C$, se tiene entonces que $(n^+, G(b))\in F$. \\
        Suponga que existe $c \in A$ tal que $(n^+, c)\in F \, \land \, c \neq
        G(b)$. \\
        Sea $F_c = F \setminus \{(n^+, c)\}$. Se tiene que $F \neq F_c$, ya que
        $(n^+, c) \in F \setminus F_c$. Además $F_c \subset F$. \\
        Observe que $F_c \in \mathcal{P}(\mathbb{N} \times A)$, además
        $(0,a)\in F$ y como $0 \neq n^+$, se tiene: $(0,a) \neq (n^+, c) \,
        \therefore \, (0,a) \in F_c$. \\
        Sea $(m, d)\in F_c$, entonces $(m,d)\in F$ y como $F \in C$, se tiene
        que $(m^+, G(d))\in F \, \land \, (m^+, G(d)) \neq (n^+, c)$; ya que de
        lo contrario, $n^+ = m^+ \, \land \, G(d) =  c \neq G(b) \, \therefore
        \, n=m \, \land \, G(d) \neq G(b)$. Así $n=m \, \land \, d\neq b$. \\
        Se tiene por lo anterior que seleccionando $n=m$, $d \neq b$. Así
        $(n,b), (n,d) \in F$ lo cual contradiría que $n \in M$. Por lo cual
        $c=G(b)$ entonces $n^+ \in M$. \\
        Los hechos 1 y 2 implican que $M=\mathbb{N}$, es decir que $\forall n
        \in N \quad \exists !b\, : \,  F(n) = b$. Así $F$ es función. \\
        Suponga ahora que $\bar{F}: \, \mathbb{N} \rightarrow S$ es una función
        tal que $\bar{F}(0) =a \, \land \, \bar{F}(n^+)=G(F(n))$. Sea $H=\{n\in
        \mathbb{N} \, | \, F(n) = \bar{F}(n) \}$. Veamos que $0 \in H$ ya que
        $F(0)=\bar{F}(0) = a$. Considere ahora a $n \in H$ (ya que es no vacío),
        entonces $F(n^+)=G(F(n))=G(\bar{F}(n))=\bar{F}(n^+)$; así $n^+ \in H$.
        Por lo cual $H = \mathbb{N}$ entonces $F=\bar{F}$.
    \end{proof}
    \begin{proposition}\label{pr:recurrencia2} Sean $A$ un conjunto, $G:\, \mathbb{N}\times A \rightarrow
        A$ una función, y $0 \in A$. Entonces existe una única función $F: \,
        \mathbb{N} \rightarrow A$ tal que:
        \begin{enumerate}
            \item $F(0) = a$
            \item $F(n^+)=G(n,F(n)) \quad \forall n \in \mathbb{N}$
        \end{enumerate}
    \end{proposition}
    Para la demostración se procede de forma similar a la prueba anterior.
    \begin{proof}
        Considere $C=\{\,T \in \mathcal{P}(\mathbb{N} \times A) \, | \, (0,a) \in
        T \, \land \, (n,b) \in T \, \Rightarrow \, (n^+, G(n,b)) \in T\,\}$.
        Trivialmente $C$ es no vacío ya que para cualquier $(n,b)\in
        \mathbb{N}\times A$,  $G(n, b) \in A$ así, $\mathbb{N}
        \times A \in C$. Considere ahora a $F = \bigcup C$; de la misma forma
        que la prueba anterior $F \in C$.
        Sea $M = \{n \in \mathbb{N} \, | \, \exists! \, b \quad (n,b) \in F$. Se
        verificará que $0 \in M$, y que si $n \in M$ entonces $n^+ \in M$.\\
        \textbf{Hecho 1.} $0 \in M$. \\
        Suponga que existe $b \in A$ tal que $b \neq a \, \land \, (0,b) \in
        F$. Sea $F_b = F \setminus {(0,b)}$ veamos que $(0,b) \in F \setminus
        F_b$ de forma que $F_b \subset F$. \\
        Considere ahora $n \in \mathbb{N}$ de forma que como $n^+ \neq 0$,
        si $(n, c)\in F_b$, entonces $(n^+,G(n,c))\in F_b$; así $F_b \in C$. 
        La condición anterior implica
        que $F \subset F_b$ así $F=F_b$ lo cual contradice que $(0,b) \in F
        \setminus F_b$, por lo tanto $a=b$, así $0 \in M$. \\
        \textbf{Hecho 2.} $n \in M \, \Rightarrow n^+ \in M$.\\
        Sea $n \in \mathbb{N} \, \land \, (n, b)\in F$. Suponga que $\exists c$
        tal que $c \neq G(n, b) \, 
        \land \, (n^+, c) \in F$. Considere ahora $F_c = F \setminus \{(n^+,
        c)\}$. Veamos que $(n^+, c) \in F \setminus F_c$ así $F_c \subset F$.
        Como $0 \neq n^+$ y $(0,a) \in F$, $(0,a) \in F_c$. \\
        Así también $(n,b) \in F_c$ ya que $n \neq n^+ \,\therefore \,
        (n,b)\neq(n^+,c)$. Y, $(n^+, G(n,b)) \in F_c$ puesto que $(n^+,
        G(n,b))\neq (n^+, c)$ así $F_c \in C$ por lo cual $F \subset F_c$. Lo
        cual contradice que $(n^+, c) \in F \setminus F_c$. Así, de los hechos
        anteriores se tiene que $0 \in M$ y si $n \in M$, entonces $n^+ \in M$.
        De lo cual se concluye que $M= \mathbb{N}$ así $F$ es función.\\ \\
        Finalmente, la unicidad se concluye de un razonamiento idéntico al de la
        proposición anterior.

    \end{proof}
    \lineskip
    \begin{proposition} Existe una única función $f: \, \mathbb{N} \times
        \mathbb{N} \rightarrow \mathbb{N}$ tal que:
        \begin{enumerate}
            \item $f(0,n) = n \quad \forall n \in \mathbb{N}$.
            \item $f(n^+,m) = f(m,n)^+ \quad \forall \, n, m \in \mathbb{N} $
        \end{enumerate}
    \end{proposition}
    \begin{proof}
        Considere la función:
        \begin{align*}
            G:& \, \mathbb{N} \times \mathbb{N} \rightarrow \mathbb{N} \\
              & \, (x,y) \rightarrow y^+
        \end{align*}
        La proposición \ref{pr:recurrencia} garantiza la existencia de una
        función $f_n:\, \mathbb{N}\rightarrow \mathbb{N} $ única tal que:
        \begin{enumerate}
            \item $f(0) = n$
            \item $f(m^+) = G(m, f(m)) \quad \forall m \in \mathbb{N}$
        \end{enumerate}
        Sea $f:\, \mathbb{N} \times \mathbb{N} \rightarrow \mathbb{N}$ la
        función definida como $f(m,n) = f_n(m)$. Se tiene así:
        \begin{enumerate}
            \item $f(0,n) = f_n(0)=n$
            \item $f(n^+,m) = f_m(n^+) = G(n, f(n))= f_m(n)^+ = f(m,n)^+$
        \end{enumerate}
        Se verificará que $f$ es única. \\
        Suponga que existe $h:\, \mathbb{N} \times \mathbb{N} \rightarrow
        \mathbb{N}$ tal que: 
        \begin{enumerate}
            \item $h(0,n) =n $
            \item $h(n^+,m) = h(m,n)^+$
        \end{enumerate}
        Sea $h_n: \, \mathbb{N} \rightarrow \mathbb{N}$ tal que $h_n(m) =
        h(m,n)$; y dado $n \in \mathbb{N}$, definamos a $M = \{\, x \in \mathbb{N}
        \, | \, h_n(x) = f_n(x) \, \}$.
        Veamos que $h_n(0)=n=f_n(0) \, \therefore \, 0 \in M$. \\
        Suponga que $x \in M$; así $h_n(x) = f_n(x)$. Se sigue entonces que:
        \[
        h_n(x^+)=h(x^+,n)=G(x,h_n(x))=G(x, f(x))= f_n(x^+)
        \]
        Así $M=\mathbb{N}$. Al ser la elección de $n$ arbitraria se tiene que
        $f(n,m)=h(n,m) \quad \forall \, n,m \in \mathbb{N}$.
    \end{proof}
    \\
    A la operación binaria dada por la proposición anterior se le denominará
    adición en $\mathbb{N}$ y $f(n,m)$ se denotará por $n + m$. 
    \vspace{1cm}

    \begin{proposition} Sean $m,n,p$ números naturales, entonces:
        \begin{enumerate}
            \item $(n+m)+p = n+(m+p)$
            \item $m + n = n + m$
            \item Si $m +p = n + p$ entonces $m = n$
            \item Si $m +p = 0$ entonces $m=0 \, \land \, n=0$
        \end{enumerate}
    \end{proposition}
    \begin{proof} 
        $\quad$ \\ 
        \begin{enumerate}
            \item Definamos a $M= \{\, x \in \mathbb{N} \, | \, x + (y + z) = (x +
                y) + z \quad \forall y, z \in \mathbb{N} \, \}$. \\
                Veamos que $0 \in M$ ya que 

                \begin{align*}
                    0 + (y + z) &= (y+z) \\
                               &= y+z \\
                               &= (0 + y ) + z \\
                \end{align*}
                Así $ 0 \in M$. \\
                Supongamos ahora que $x \in M$, es decir, que 
                \[
                    x (y+z) = (x+y)+z
                \]
                Entonces se tiene que:
                \begin{align*}
                    x^+ + (y +z) & = (x + (y + z))^+ \\
                                 & = ((x + y)+ z))^+ \\
                                 & = ((x+ y)^+ +z) \\
                                 & = (x^+ + y ) + z
                \end{align*}
                Así $M = \mathbb{N}$.
            \item Definamos a $M = \{\, x \in \mathbb{N} \, | \, x + y = y + x
                \quad \forall \, y \in \mathbb{N} \, \}$.
                \begin{itemize}
                    \item Se verificará que $0 \in M$.
                 \begin{itemize}
                    \item Caso 1: $y=0$. \\
                        En dado caso se tiene:
                        \begin{align*}
                            y+0 &= 0 + 0 \\
                                &= 0 + y
                         \end{align*}
                    \item Caso 2: $y \neq 0$. \\
                        En dado caso $\exists y_-$ tal que $y_-^+ = y$
                         \begin{align*}
                            0+y &= y \\
                            y+0 &= (0 + y_-)^+ \\
                                &= y_-^+ \\
                                &= y
                        \end{align*}
                \end{itemize} 
                Así, verificamos que $0 \in M$.
            \item Ahora verificaremos que si $x \in M \, \Rightarrow \,x^+ \in
                M$. \\

            Sea $M' = \{ \, x \in \mathbb{N} \, | \, x^+ + y = x + y^+ \, \}$. \\
            \begin{itemize}
                \item Se probará que $0 \in M'$. \\
                    Veamos que $\forall y \in \mathbb{N}$:
                    \begin{align*}
                        0^+ + y &= 1 + y \\
                                &= (y + 0)^+ \quad \mathrm{Como}\, 0 \in M\\
                                &= (0 + y)^+\\
                                &=  y^+ + 0 \quad \mathrm{Adem\acute{a}s}\\
                        y^+ + 0 &=  0 + y^+ = 0
                    \end{align*}
                    Así $0 \in M'$.
    \begin{remark} Ésto último prueba que $1 + y = y^+$.
    \end{remark}
                    \item Se probará que si $x \in M'$ entonces $x^+ \in M'$ \\
                        $\forall y \in \mathbb{N}$
                        \begin{align*}
                            x^+ + y &= (y + x)^+ \\
                                    &= (y + x) + 1 \\
                                    &= y + (x + 1) \\
                                    &= y + (x + 0^+) \\
                                    &= y + (x^+ + 0) \\
                                    &= y + (0 + x^+) \\
                                    &= y + x^+
                        \end{align*}
                        Así $x^+ \in M'$.
                        Por lo tanto $M' = \mathbb{N}$.
            \end{itemize}

                Suponga que $x \in M$, entonces $x + y = y +x \quad \forall y
                \in \mathbb{N}$. Así:
                \begin{align*}
                    x^+ + y &= (y + x)^+ \\
                            &= (y + x)+1 \\
                            &= y + (x + 1) \\
                            &= y + (0 + x^+) \\
                            &= y + x^+
                \end{align*}
                Así $M = \mathbb{N}$
        \end{itemize}
    \item Sea $M = \{ p \in \mathbb{N} \, | \, (m + p = n + p \, \Rightarrow \, m
        = p) \quad \forall \, n, m \in \mathbb{N} \, \}$. 
        \begin{itemize}
            \item $0 \in M$. \\
                Veamos que $\forall \, m, n \in \mathbb{N}$:
                Suponga que $ m + 0 = n + 0 $
                \begin{align*}
                    m + 0 &= n + 0 \\
                    0 + m &= 0 + n \\
                        m &= n 
                \end{align*}
                Así $0 \in M$ 
            \item $\forall \, m, n \in \mathbb{N}$ suponga que $(p +m = p + n) \,
                \Rightarrow \,m = n 
                \quad p \in \mathbb{N}$. Vemos que si:
                \begin{align*}
                    p^+ + m &= p^+ + n \\
                    (m + p)^+ &= (n + p)^+ \\
                    (p + m)^+ &= (p + n)^+ \, \Rightarrow \\
                    p + m &= p + n \, \Rightarrow \\
                        m &= n 
                \end{align*}
        \end{itemize}
        Así $M = \mathbb{N}$.
    \item Sea $M= \{\, x \in \mathbb{N} \, | \, x + y = 0 \, \Rightarrow \, x =
    0 \, \land \, y = 0 \, \quad \forall x \in \mathbb{N} \,\}$. Considere a $x,
    y \in \mathbb{N}$ tales que $x + y = 0$. Entonces:
        \begin{align*}
            x + y &= 0 \\
                  &= 0 + y \\
        \end{align*}
        De la propiedad 3, tenemos que $x = 0$. Como $x, y$ fueron arbitrarios: 
        $x \in M$ entonces $\mathbb{N} \subset M$. 
        Así $M = \mathbb{N}$. 
    \end{enumerate}
    \end{proof}
    \begin{proposition} Existe una única función $g: \, \mathbb{N} \times
        \mathbb{N} \, \rightarrow \mathbb{N}$ tal que:
        \begin{enumerate}
            \item $g(0,n) = 0\quad \forall n \in \mathbb{N}$
            \item $g(n^+,m) = g(n,m)+m \quad \forall n,m \in \mathbb{N}$
        \end{enumerate}
    \end{proposition}
    \begin{proof}
        La proposición \ref{pr:recurrencia} implica que existe una única función
        $F_n: \mathbb{N} \, \rightarrow \, \mathbb{N}$ así se tiene que la 
    \end{proof}

\begin{definition}
    Sea $x \in \mathbb{N}$. Se define el conjunto de descendientes de $x$
    denotado por $Dx$ como el conjunto $\{ \, S^n (x) \, | \, n \in
    \mathbb{N}\,\}$.

    \begin{remark} $x\in Dx$.
    \end{remark}
\end{definition}
\begin{definition} Sea $A \subset \mathbb{N}$. Se define el conjunto de
    sucesores de elementos de $A$, denotado por $A^{\Delta}$, como el conjunto
    $\{S(x) \, | \, x \in A \, \}$.
\end{definition}
\begin{proposition}
    \label{prop:21}
    Sean $x$, $y \in \mathbb{N}$, entonces: 
    \begin{align*}
        &i)\, Dx = \{x\} \cup Dx^+ 
        &ii)\,Dx^+ \subset [Dx]^{\Delta}\\
        &iii)\,x \notin Dx^+  &iv)\,Dx = Dy \, \mathrm{entonces} \, x=y \\
    \end{align*}
\end{proposition}
\begin{proof}$i)$ Sea $u \in Dx$, entonces $\exists n \in \mathbb{N}$ tal que $u =
    S^n(x)$. Entonces $u=x \, \lor \, u \neq x$. \\
    Si $u = x$, entonces $u \in {x} \cup Dx^+$. \\
    Si $u \neq x$, entonces $n \neq 0$. Así, como $S(\mathbb{N}) = \mathbb{N}
    \setminus \{0\} $se tiene $\n \in S(\mathbb{N})$, por lo tanto $\exists m \in
    \mathbb{N}$ tal que $n= S(m)$, \textit{i.e.}, $n=m^+ \, \therefore u = S^{m^+}(x) = 
    S\circ S^{m}(x) = S^m \circ S(x) = S^m(x^+)$ lo cual muestra que $u\in
    Dx^+\, \therefore \, u \in \{x\} \cup Dx^+$. Así se demuestra $i)$. \\

    $ii)$ Sea $m \in Dx^+$. Se debe verificar que $m \in [Dx]^{\Delta}$. Como $m
    \in Dx^+$, existe $n\in \mathbb{N}$ tal que: $m = S^n(x^+) =S^n \circ S(x) =
    S \circ S^n (x)$, lo cual muestra que $m \in [Dx]^{\Delta}$. \\

    $iii)$ Sea $M= \{x \in \mathbb{N} \, | \, x \notin Dx^+ \}$ \\
    Si $0 \in D0^+$, entonces $\exists r \in \mathbb{N}$ tal que: \\
    $0 = S^r(0^+) = S^r \circ S(0) = S \circ S^r (0)$, lo cual implica que $0
    \in S(\mathbb{N})$ en contradicción con el hecho $S(\mathbb{N}) =
    \mathbb{N}\setminus \{0\}$ por lo tanto $0 \notin D0^+$, i. e., $0 \in M$.
    \\
    Sea $x \in M$. Se verificará que $x^+ \in M$. \\
    Suponga que $x^+ \notin M$, entonces $x^{++}\in D0^{++}$. De $ii)$ $Dx{++}
    \subset [Dx^+]^{\Delta}\, \therefore \, x^+ \in [D_x^+]^{\Delta}$. Así
    $\exists q \in Dx^{+}$ tal que $x^+ = S(q)$, es decir, $x^+ = q^+ \,
    \therefore \, x=q$, y así $x \in Dx^+$, lo cual contradice que $x \in M\,
    \therefore \, x^+ \in M$. Así $M = \mathbb{N}$. \\

    $iv)$ Por hipótesis $Dx=Dy$. Suponga que $x \neq y$. Observe que, como $x
    \in Dx$ y $Dx = Dy$, se tiene que $x \in Dy$. Además $Dy = \{y\} \cup Dy^+$.
    Así como $x \neq y$, se tiene $x \in Dy^+\, \therefore \, Dx \subset Dy$ (ya
    que si $u \in Dx, \, \exists n \in \mathbb{N}$ tal que $u = S^n(x)$. Además,
    como $x \in Dy^+ \, \exists m \in \mathbb{N}$ tal que $x \in S^m(y^+)\,
    \therefore \, u=S^n \circ S^m(y^+)$ lo cual muestra que $u \in Dy^+$). Así
    $Dy \subset Dy^+ \, \therefore \, y \in Dy^+$ lo cual contradice a $iii)$.
    Así $x=y$.
\end{proof}
\begin{proposition}\label{prop:22} Sea $R= \{\, (m,n)\in \mathbb{N}\times \mathbb{N} \, | \, n
    \in Dm\}$. $R$ es un orden parcial en $\mathbb{N}$.
\end{proposition}
\begin{proof}
    $1)$ Dado $n \in \mathbb{N} \, n \in Dn\, \therefore (n,n)\in R$. Es decir, $R$
    es reflexiva. \\
    $2)$ Sean $(n,m)$ y $(m,r)\in R$, entonces $m \in Dn \, \land \, r \in Dm\,
    \therefore \, Dm \subset Dn \, \land \, Dr \subset Dm \, \therefore \, Dr
    \subset Dn$ entonces $r \in Dn \, \therefore \, (n,r) \in R$.
    $3)$ Suponga que $(n,m)$ y $(m,n)\in R$. Entonces $m \in Dn \, \land n \in Dm
    \, \therefore \, Dm \subset Dn \, \land \, Dm \subset Dn \, \therefore \, Dm =
    Dn$, y de la proposición anterior se tiene que $m=n$. Así $R$ es antisimétrica.
\end{proof}
\begin{lemma} 
    Sea $x\in \mathbb{N}$, entonces:
    \[
        Dx = \bigcap \{\, A \in \mathcal{P}(\mathbb{N})\, | \, x \in A \, \, n
        \in A \,\Rightarrow \, n^+ \in A\}
    \]
\end{lemma}
\begin{proof}
    Sea $B = \bigcap\{ \, A \in \mathcal{P}(\mathbb{N}) \, | \, x \in A \,\land \,
    \forall n \in \mathbb{N} \, n \in A \, \Rightarrow \, n^+ \in A \}$. Sea $M
    = \{\, n \in \mathbb{N} \, | \, S^n(x) \in B\,\}$. Sea $Z =\{ \, A \in \mathcal{P}(\mathbb{N}) \, | \, x \in A \,\land \,
    \forall n \in \mathbb{N} \, n \in A \, \Rightarrow \, n^+ \in A \}$.
    Entonces $B = \bigcap Z$. Observe que $x \in A \, \forall A \in Z \,
    \therefore \, x \in \bigcap Z $, es decir $x \in B$, \textit{i.e.}, $ S^o(x) \in B \,
    \therefore \, 0 \in M$. \\
    Sea $n \in M$, entonces, $ S^n(x) \in B \, \therefore \, S^n^+(x) = S^n\circ
    S(x) = S \circ S^n(x) \in B$. Como $B \in Z \, \land \, S^n(x) \in B$, se
    tiene que $S\circ S^n(x) \in B \, \therefore \, S^n^+(x) \in B \, \therefore
    \, n^+ \in M$. Así $M=\mathbb{N}$. En consecuencia $Dx \subset B$. \\
    se tiene que $x \in Dx$. Además $u \in Dx$ implica $u^+ \in Dx$. Por lo
    tanto $B \subset Dx$. Así $B = Dx$.
\end{proof}
\begin{proposition}\label{prop:23}
    Sea $M \subset \mathbb{N}$, $M \neq \emptyset$. Si $x^+ \in M \, \forall x \in M$,
    entonces existe un único $n \in \mathbb{N}$ tal que $M=Dn$.
\end{proposition}
\begin{proof}
    \textbf{Hecho 1:} Sea $K=\{ n \in \mathbb{N} \, | \, n \notin M \,
    \Rightarrow M \subset Dn^+\}$.\\
    Por el lema anterior $D0 = \bigcap \{A \in
        \mathcal{P}(\mathbb{N}) \, | \, 0 \in A  \, \land \, (\forall n \in
    \mathbb{N} \, n \in A \Rightarrow n^+ \in A)\,\}$. Se tiene entonces $0
    \in D0 \, \land \, (n \in D0 \Rightarrow n^+ \in D0) \, \therefore \, D0
    = \mathbb{N}$. Por otra parte $D0 = \{0\} \cup Do^+$. Así $\mathbb{N} =
    \{0\} \cup D0^+$. En consecuencia $\mathbb{N}\setminus \{0\} \subset
    D0^+$. \\
    Así mismo como $0 \notin D0^+$. y $D0^+ \subset \mathbb{N}$. Se tiene
    $D0^+ \subset \mathbb{N} \setminus \{ 0\}\, \therefore \, D0^+ =
    \mathbb{N} \setminus \{0\}$ y así si $0 \notin M$, se tiene $M \subset
    D0^+\, \therefore\, 0 \in K$. \\
    Sea $n \in K$, se verificará que $n^+ \in K$.\\
    Suponga $n^+ \notin M$, entonces $n \notin M$ ya que $x^+ \in M\,
    \forall x \in M$. Así como $n \in k$, se tiene $M \subset Dn^+$, y como
    $Dn^+ = \{n^+\} \cup Dn^{++}$. Además $n^+ \notin M$, luego $M \subset
    Dn^{++}$, lo cual implica que $n^+\in K\, \therefore \, K
    =\mathbb{N}$.\\
    \textbf{Hecho 2:} $\forall n \in \mathbb{N} \, [(n \notin M \, \land \,
    n^+ \in M)\Rightarrow M=Dn^+]$.\\
    Sea $n \in \mathbb{N}$ tal que $n\notin M \, \land \, n^+ \in M$. Como $n
    \notin M$, por el hecho 1 se tiene que $M \subset Dn^+$. Por el lema
    anterior $Dn^+ \subset A\quad \dorall A$ tal que $ n^+ \in A \, \land \,
    \forall x \in \mathbb{N} \, (x \in A \Rightarrow x^+ \in A )$. Como
    $n^+ \in M \, \land \, x^+ \in M \quad \forall x \in M$, se concluye que
    $Dn^+ \subset M \, \therefore \, Dn^+ = M$. \\

    $0 \in M \, \lor \, 0 \notin M$ \\
    - Si $0\in M$, entonces se tiene que $M = \mathbb{N}$ y así $M=D0$.\\
    - Suponga que $0\notin M$.\\
    Se verificará que existe $n \in \mathbb{N}$ tal que $n \notin M \, \land \,
    n^+ \in M$,de esta manera, por el hecho 2 se tendrá que $M=Dn^+$. \\
    Procediendo por reducción al absurdo suponga que no existe tal $n \in
    \mathbb{N}$. Entonces: $\forall n \in \mathbb{N}  \, \quad n \in M \, \lor
    \, n^+ \notin M\, \ldots (\lambda)$. \\
    Considere el conjunto $T = \{n\in \mathbb{N} \, | \, n \notin M\}$.\\
    -Se tiene $0 \in T$ ya que se está considerando el caso en que $0 \notin M$.
    \\
    -Sea $n \in T$, entonces $n \notin M\, \therefore$ de $(\lambda)$ se tiene
    $n^+ \notin M \, \therefore \, n^+ \in T$. Teniendo así $T = \mathbb{N}$,
    por lo cual $M = \emptyset$, lo cual contradice que $M \neq \emptyset$. Por
    lo tanto sí existe $n \in \mathbb{N}$ tal que $n \notin M \, \land \, n^+
    \in M$ y por ello, del hecho 2: $M=Dn^+$. De la proposición \ref{prop:21} se
    tiene que si $M=D^n \, \land M = D^m$, entonces $n=m$, así $n$ es único.
\end{proof}

\begin{proposition}\label{prop:24} $R = \{(m,n) \in \mathbb{N} \times
    \mathbb{N} \, | \, n \in Dm \}$ es un buen orden en $\mathbb{N}$.
\end{proposition}
\begin{proof}
    Se ha probado previamente que $R$ es un orden parcial. Sea $A \subset
    \mathbb{N}$, $A \neq \emptyset$. Se debe probar que $A$ tiene un primer
    elemento. Considere el conjunto $B = \{ R \in \mathcal{P}(\mathbb{N}) \, |
    \, A \subset T \, \land \, S(T) \subset T \}$; veamos que $B \neq
    \emptyset$, ya que $ \mathbb{N} \in B$. \\
    Sea $M = \bigcap B$, observe que $M \subset B$. Como $A \subset M \, \land
    \, A \neq \emptyset$, se tiene $M \neq \emptyset$. Además, como $M \in B \,
    \forall x \, ( x \in M \Rightarrow x^+ \in M)$. La proposición \ref{prop:23}
    implica que existe un único $n \in \mathbb{N}$ tal que $M = Dn$. Se
    verificará que dicho $n \in A$. Procediendo por contradicción suponga que
    $n \notin A$. Se tiene por la proposition \ref{prop:21} que $Dn = \{n\} \cup
    Dn^+$. Como $A \subset M$, entonces $A \subset \{n\} \cup Dn^+$, y así como
    $n \notin A$, se tiene: $A \subset Dn^+ \ldots (i)$.\\
    Además, $\forall x \, (x \in Dn^+ \Rightarrow x^+ \in Dn^+)$, es decir,
    $S(Dn^+)\subset Dn^+ \ldots (ii)$. \\
    $i)$, y $ii)$ implican que $Dn^+ \in B \, \therefore \, M \subset Dn^+$,
    \textit{i.e.}, $Dn \subset Dn^+$. Así, como $n \in Dn$, se tiene $n\in Dn^+$, lo cual
    contradice la parte $iii)$ de la proposición \ref{prop:21} $\therefore \, n
    \in A$.\\
    Se mostrará que $n$ es el primer elemento de $A$.\\
    Sea $x \in A \, | \, x=n \, \lor \, x\neq n$.\\
    $a)$ Si $x=n$, se tiene $(n,x) \in R$, ya que $R$ es en particular
    reflexiva.
    $b)$ Suponga que $x \neq n$. \\
    Como $A\subset M \, \land \, M=Dn$, se tiene $A\subset\{n\}\cup Dn^+$. Yasí,
    dao que $x\neq n$, se tiene $x \in Dn^+\, \therefore (n^+,x)\in R$. Por otra
    parte $(n,n^+)\in R$ ya que $n^+ \in Dn$. Y, como $R$ es transitiva, se
    concluye que $(n,x)\in R$, finalmente mostrando que $n$ es el primer
    elemento de $A$.
\end{proof}
\textbf{Notación:} $(n,m)\in R$ se denotará $n \leq m$. Si, además $n \neq m$,
se escribirá $n < m$.
\begin{proposition} \label{prop:25} Sean $n,m \in \mathbb{N}$. Entonces se
    cumple una única de las siguientes tres afirmaciónes:
    \[
        i) n=m \quad \quad ii) n<m \quad \quad iii) m<n
    \]
\end{proposition}
\begin{proof}
    Como $R$ es un buen orden en $\mathbb{N}$, se tiene que $R$ es un orden
    total en $\mathbb{N}$. Así $(n,m)\in R \, \lor \, (m,n) \in R$. Así, como
    $R$ es antisimétrica, si $n \neq m$, solo se puede cummplir una única de las
    siguientes condiciones: $(n,m)\in R$ o $(m,n)\in R$, así se tiene únicamente
    una de $n<m$, $m<n$.\\
    Si $n=m$, no se cumple ni $n<m$ ni $m<n$. 
\end{proof}
\begin{proposition}\label{prop:26} Sean $n, m \in \mathbb{N}$, si $n^+ >m$,
    entonces $n \geq m$. 
\end{proposition}
\begin{proof}
    Por hipótesis $n^+ > m$. Procediendo por contradicción, suponga que no se
    cumple $n \geq m$, entonces $(m,n)\notin R\, \therefore \, m \neq n \, \land
    \, \neg (m<n)$. La proposición \ref{prop:25} implica que $n<m  \, \querefore
    \, n \leq m \, \land n \neq m$. En particular $(n,m)\in R$. Así $m \in Dn \,
    \therefore \, \exists r \in \mathbb{N}$ tal que $m = S^r(n)$. \\
    Como $m \neq n $, se tiene $r \neq 0\, \therefore \, r \in \mathbb{N}
    \setminus \{0\}$, \textit{i.e.}, $r \in S(\mathbb{N})$, por lo cual $\exists q \in
    \mathbb{N}$ tal que $r = q^+$. \\
    Así $m= S^{q^+}= S^q \circ S(n) = S^q(n^+)$ lo cual implica que $m \in Dn^+
    \, \therefore \, (n^+,m) \in R \ldots (\lambda)$. Por hipótesis se tiene
    $n^+ > m \ldots (\alpha)\, \therefore \, n^+ \neq m$. De $(\lambda)$ se
    tiene $n^+ < m \ldots (\beta)$. De $(\alpha)$ y $(\beta)$ se tiene $m < n^+
    \, \land n^+ < m$, lo cual contradice la proposición \ref{prop:25}.
\end{proof}
\begin{proposition} \label{prop:27} 
    $i) \forall  n,m \in \mathbb{N} \, \n \leq m$ sí y sólo sí $\exists p \in
    \mathbb{N}$ tal que $n + p=m$. \\
    $ii) \forall n,m,p \in \mathbb{N} \, n <m$ sí y sólo sí $n +p < m +p$. \\
    $iii) \forall n,m,p \in \mathbb{N}$, si $p \neq 0$, entonces $n <m$ sí y
    sólo sí $np<mp$. 
\end{proposition}
\begin{proof} \\ 
     
    $i)$ Dados $n,m \in \mathbb{N}$, considere el conjunto $K=\{ t \in
    \mathbb{N} \, | \, S^t(n) = n+t\}$.\\
    $\cdot$ $S^0 (n) = n = n +0 \, \therefore \, 0 \in K$. \\
    $\cdot$ Sea $t \in K$, entonces $S^{t^+} = S \circ S^t(n) = S(S^t(n)) =
    S(n+t) = S(t +n) = (t + n)^+ = t^++ n = n + t^+$. Se tiene así que $t^+ \in
    K\, \therefore \, K = \mathbb{N}$.  \\
    $\Rightarrow )$ Por hipótesis $n \leq m$, \textit{i.e.},  $m \in Dn \,
    \therefore \, \exists p \in \mathbb{N}$ tal que $m = S^p (n)$. Como $K =
    \mathbb{N}$, $p \in K\, \therefore \, S^p(n) = n +p$ \textit{i.e.},
    $m=n+p$.\\
    $\Leftarrow )$ Ahora por hipótesis, $\exists p \in \mathbb{N}$ tal que $n +
    p = m$. Como $p \in \mathbb{N} \, \land K = \mathbb{N}$, se tiene que $p \in
    K\, \therefore \, n+p=S^p(n)$, \textit{i.e.} $m \in Dn \, \therefore  \, n
    \leq m$.\\

    $ii)$ Sean $n,m,p \in \mathbb{N}$. \\
    $\Rightarrow ) $ Suponga que  $n <m$, entonces $(n,m)\in R \, \land \, n
    \neq m$. En particular $m \in Dn\, \therefore \, \exists r \in \mathbb{N}$
    tal que $m = S^r(n)$, \textit{i.e.}, $m = n + r$. Observe que $r\neq 0$, ya
    que $ n \neq m$. \\
    Como $m = n + r$, se tiene: \\
    $m + p = (n + r)+p= (n+p)+r ) S^r(n +p)\, \therefore \, m + p \in D_{n+p}$.
    En consecuencia $n + p \leq m +p$. Además, como: $m +p = (n +p)+r$ y $r \neq
    0$ se tiene de $A_3$ que $m+p \neq n+p\, \therefore \, n+p < m+p$.\\
    $\Leftarrow )$ Por hipótesis $n+p < m+p$. \\
    Suponga que no se cumple $n <m$. Entonces $n=m \, \lor \, m<n$. En el primer
    caso $n + p =m + p$, lo cual no ocurre. En el segundo caso, de la
    suficiencia probada previamente, se tendría: $m + p < n + p$, lo cual no
    ocurre. $\therefore\ n < m$.\\

    $iii)$ Sean $n, m, p \in \mathbb{N}, p \neq 0$.\\
    $\Rightarrow )$ Suponga que $n <m$. Entonces, de $i)$ $\exists q \in
    \mathbb{N}$ tal que $m = n + q$. Además $q \neq 0$ ya que $n \neq m \,
    \therefore \, mp = (n +q)p = np + qp$.\\
    $M_4$ implica que $qp \neq 0$ ya que $q \neq 0 \, \land \,  p \neq 0 \,
    \therefore \, mp > np$. \\
    $\Leftarrow )$ Asuma ahora $np <mp$, se debe probar $n <m$. \\
    Suponga que no se cumple $n <m$. Entonces $n = m \, \lor \, m<n$. En el
    primer caso se tendría $np = mp$, lo cual no ocurre. En el segundo caso se
    tendría, por la suficiencia previamente probada, $mp < np$ lo cual no ocurre
    $\therefore\, n<m$.
\end{proof}
\subsection{Ejercicios}
\begin{problem}
    
    $a)$ Sea $m \in \mathbb{N}$ Demuestre que existe una única función $F_m:
    \mathbb{N} \rightarrow \mathbb{N}$ tal que $F_m(0) = 1$, $F_m(n^+) =
    g_m(F_m(n))$, siendo $g_m$ la función: $g_m: \mathbb{N} \rightarrow
    \mathbb{N}\quad n \rightarrow nm$ \\
    $b)$ Considere la función $f: \mathbb{N}\times \mathbb{N} \rightarrow
    \mathbb{N}\quad (m,n) \rightarrow F_m(n)$. \\
    Verifique que:
    \begin{enumerate}
        \item $f(m,0) =1, \forall m \in \mathbb{N}$ 
        \item $f(m,n^+) = F_m(n)m$
    \end{enumerate}
    Denotando $f(m,n)$ por $m^n$, verifique lo siguiente: \\
    Dados $m,n, u \in \mathbb{N}$
    \begin{enumerate}
        \item $u^m \cdot u^n = u ^{m+n}$
        \item $(u^m)^n = u^{mn}$
        \item $(un)^m = u^mn^m$
        \item $1^n = 1$
    \end{enumerate}
\end{problem}
    \section{Equivalencias del Axioma de Elección}
    \begin{prop12}
Sea $ (N,R) $ un conjunto parcialmente ordenado, tal que toda cadena contenida en $ N $, tiene un supremo en N. 
Sea $ f:N\rightarrow N $ una funcion que satisface:
\[ \forall x \in N   \ \ \  (x,f(x)) \in R \]
Entonces existe $ n \in N $ tal que $ f(n) = n $.


\end{prop12}




\begin{proof}
Dado $ u \in  N  $ se dice que $ A \subset N$ es admisible con relación a $ u $ si cumple:
\begin{itemize}
\item[\textit{i)}] $ u \in A$
\item[\textit{ii)}]$ f(A)\subset A $
\item[\textit{iii)}] Si B es una cadena contenida en A, entonces en A  mismo existe el \textit{supremo} de B
\end{itemize}

Sea $ M $ el conjunto cuyos elementos son todos los subconjuntos  admisibles conrespecto a u de N y sea:
$S= \bigcap\limits_{x\in M} x$.
\begin{proof}

\begin{itemize}

\item[a)]
 $$ v \in S  \ \ \ \  ya \ \ que  \ \ \ \  u \in x \  \ \ \forall x \in M $$
 
\item[b)]
Sea $ t \in S$
$$ t \in x \ \ \  \forall \ x\in M$$
$$f(t) \in x \ \ \ \forall \ x\in M $$
Lo cual muestra que $ f(S) \subset S $

\item[c)] Sea B una cadena contenida en S
$$ B \subset x \ \ \ \forall x \in M $$
$$ \therefore \ \ \ Sup(B) \in S$$ 

\end{itemize}
a), b) c) muestra que S es admisible con relación a u
\end{proof}
Observe que: si  $ S' \subset S      \ \ \wedge \ \ S' $ es u-admisible entonces $ S' = S $ 


\bigskip
\bigskip

\textbf{Hecho 1:} u  es el primer elemento de S,
 i.e.:$$ (u,x) \in R \ \ \forall \ x \in S $$
 
\begin{proof}
Sea $ A = \lbrace x \in S \ \ \ \vert (u,x) \in R \rbrace $

  

Para probar el hecho anterior  se debe probar que  $ A = S $. \linebreak 
Como $ A \subset S $, por  la observacion anterior basta con verificar que A es u-admisible.
\begin{itemize}
\item[\textit{i)}] Como $ R $ es en particular reflexiva  $(u,u) \in R$. Además $ u \in S $ ya que S es  u-admisible.
\item[\textit{ii)}] Sea $ x \in A $, se verificara que $ f(x) \in A $
\bigskip
Como $ x \in A $,  $ (u,x) \in R $. Además $ (x,f(x)) \in  R $ Por hipostesis de la proposioción. \\

$ \therefore $  siendo $ R $ transitiva se tiene $ (u,f(x)) \in R $ \\

Como $ x \in A \ \ \ \wedge \ \ \ A \subset S $,  $ x \in S $  y por ser S u-admisible $ f(x) \in S $
$\therefore f(x) \in A $.   En consecuencia $ f(A) \in A $

\item[\textit{iii)}] Sea B una cadena contenida en A. Por hipotesis, $ supB $ existe  en N.\\
Como $ A \subset S $ se tiene a $ B \subset S $.\\
Así siendo S u-admisible se tiene $ supB \in S $\\
$$ (t,supB) \in R \ \ \ \forall t \in B $$
Además como $ B \subset A  $ se tiene  que $ (u,t) \in R \ \ \ \forall t \in B $.
$$ \therefore \ \ \ (u,supB) \in R \ \ \ \ \therefore supB \in A $$.

\end{itemize}  
\textit{i), ii), iii)} muestran que A es u-admisible  
\end{proof}
Dado $ x \in  $, P(x) denotara la propiedad $$ [y \in S \diagdown {x} \wedge (y,x) \in R] $$

\textbf{Hecho 2:}   Si   $ \overline{x} \in S \ \ \ \ \wedge \ \ \ \ P(\overline{x}) $  entonces  $ \forall
z \in S \ \ \ \ \ (z,\overline{x}) \in R \vee (f(\overline{x}),z) \in R $

\begin{proof}
Sea $$ B =  \lbrace z \in S \vert (z,\overline{x}) \in R \vee (f(\overline{x}),z) \in R \rbrace $$
 Se debe probar que $B=S$.\\
Como $B \subset S$, basta con probar que B es u-admisible
\begin{itemize}
\item[\textit{i)}] Como  $ u \in S \ \ $ puesto que S es u-admisible\\
Como $u$ es el  primer elemento de S y $ \overline{x} \in S $ se tiene $ (u, \overline{x}) \in R $
$$ \therefore \ \ \ \ u \in B $$
\item[\textit{ii)}] Sea $x \in B$ se verificará que $ f(x) \in B $\\
Como $ x \in B \ \ \ \wedge \ \ \  B \subset S $, se tiene $x \in S$ $ \therefore$ por ser S u-admisible se tiene $f(x) \in S $ 
$$ x = \overline{x} \ \ \vee \ \  x \neq \overline{x} $$
\begin{itemize}
\item[a)] Si $ x = \overline{x} $,  entonces $ f(x) = f(\overline{x}) $
$$ \therefore \ \ (f(x), f(\overline{x})) \in R $$
y así $f(x) \in B$
\item[b)]
Considere $ x \neq \overline{x} $ \\
Como $x \in B$ se tiene $(x,\overline{x}) \in R \ \  \vee \ \ (f(\overline{x}),x) \in R$.  \\
\begin{itemize}
\item[I)] Si $ (x,\overline{x}) \in R $, como $ x \neq \overline{x} $ y además se tiene $ P(\overline{x}) $,  entonces $ (f(x),\overline{x}) \in R $\\
 lo cual  muestra que $ f(x) \in B $
\item[II)] Suponga que $ (x,f(x)) \in R $ se tiene que $ (f(x), f(\overline{x}) \in R $ \\
$ \therefore \ \ \ \  f(x) \in B $

\end{itemize}

\end{itemize}
Se ha probado que $ f(B) \subset B $

\item[\textit{iii)}] Sea $ F \subset B $ una cadena, F es entonces ena cadena en S,\\
 entonces  $ \therefore \ \ \ F \in S  $\\
 Como $ F \subset  B $\\
 $ \forall z \in  F $ se tiene $ (z, \overline{x}) \in R \ \ \ \vee \ \ \ (f(\overline{x}), z) \in R $
 $$ \therefore \ \ \  (\forall z \in F \ \  (z,\overline{x}) \in R \ \ \vee \ \  (\exists z \in F \ \ \pitchfork (f(\overline{x}),z) \in R ) $$
 
\begin{itemize}
\item[a)]
Suponga que $ \forall z \in F (z,\overline{x}) $ entonces $ \overline{x} $ es una cota superior de F\\
$ \therefore \ \ \ (supF, \overline{x}) \in R $ lo cual implica $ supF \in B $
\item[b)] Suponga ahora que $ \exists z \in F tal que (f(\overline{x}), z) \in R $ \\
Como $$ z \in F \ \ \ \ (z,supF) \in R \ \ \ \therefore (supF,f(\overline{x}) \in R $$
Por lo cual $ supF \in B $ 
\end{itemize}

\end{itemize}
\textit{i), ii), iii)} muestran que B es u-admisible  $ \therefore \ \ \ B=S $.
\end{proof}
\textbf{Hecho 3:}  $$\forall x \in S \ \ \ \  se \ \ \  tiene \ \ \ \  P(x) $$
\begin{proof}
Como $ C \subset S $ basta probar que C es u-admisible 
\begin{itemize}
\item[\textit{i)}] Como S es u-admisible, $ u \in S $ se verificara P(u).\\
Suponga $ \neg P(u) $\\
Entonces $ \exists y \in S \diagdown \lbrace u \rbrace $ tal que $ (y,u) \in R $\\
Como $ y\in S  $ y $ u $ es el primer elemento de S, se tiene $(u,y) \in R  $. Dado que $ (y,u) \in R $ 
se tiene $ u=y $ (por ser R un orden parcial) lo cual contradice 
$$ y \in S \diagdown \lbrace u \rbrace $$ 
Lo cual implica $ u \in C $
\item[\textit{ii)}] Se verificara ahora que $ f(C) \subset C $
Sea $ x \in C $, se debe probar que  $ f(x) \in C $\\
Como $ x \in C  \ \ \wedge \ \ C \subset S $ se tiene $ x \in S $ y asi, por ser S u-admisible \\
Se tiene $ f(x) \in S $ .\\
 Para concluir que $f(x) \in C  $ basta probar que $ P(f(x)) $\\
Asuma que $ y \in S \diagdown \lbrace f(x) \rbrace $ tal que $ (y,f(x)) \in R $\\
Se verificara que $ (y,f(x)) \in R $\\
Como $ x \in C $ se tiene $ x \in S \ \ \ \wedge \ \ \ P(x) $\\
dado que $ y \in S $, y el \textbf{Hecho 2}, se tiene $ (y,x) \in R \ \ \ \vee \ \ (y,f(x)) \in R $\\

\newpage

Observe que $ \neg ((f(x),y) \in R)  $ ya que si $ (f(x),y) \in R $, como\\
 $ (y,f(x)) \in R $ se tendría $ y=f(x) $ lo cual contradice que $ y \in S \diagdown \lbrace f(x) \rbrace $ 
 $$\therefore \  \  \ (y,x) \in R $$
$$ y=x \ \ \vee y \neq x $$
\begin{itemize}
\item[a)] Si $ y=x $, $ f(y)=f(x)$ Por lo cual $(f(y),f(x)) \in R  $
\item[b)] Considere ahora $ y \neq x $, entonces $$ y \in S \diagdown \lbrace x \rbrace \ \ \ (y) $$
por lo cual, dado $ P(x) $ se concluye que $ (f(y), x) $. Asi mismo  se tiene que $ (f(y),x) \in R $ \\
por transitividad se tiene  $ (f(y),f(x)) \in R $
\end{itemize}
 \item[\textit{iii)}] Sea $ F \subset C $  y  una cadena. Se probara que $ supF \in C $ \\
 Como $ C \subset S $ y $ F \subset S $ por lo cual siendo S u-admisible se tiene\\
  $$  supF \in S $$ \\
 Así para probar que $ supF \in F $ basta verificar que se cumple $ P(supF) $\\
 Sea $  w = supF $ \\
 Sea $ y \in S \diagdown \lbrace w \rbrace $ tal que $ (y,w) \in R $. Se debe probar $ (f(y),w) \in R $
 Observe $ \exists y_{1} \in F $ tal que $ (y, y_{1}) $\\
 (Ya que en caso contrario dado cualquier elemento $ y_{1} \in F $ se tiene $ \neg (y,y_{1}) $ Además como $ F \subset C $ ; $ y_{1} \in C $  y así $ P(y_{1}) $ y por el \textit{Hecho 2}, como $ y_{1} \in S \ \ \ \wedge P(y_{1}) $ y además  y $ y \in S $, necesariamente 
 $$ (y,y_{1})\in R \ \ \ \vee \ \ \  (y,f(y_{1})) \in R $$ 
 Así como se tiene $ \neg ( (y,y_{1})\in R) $ se tendría $ (f(y_{1}),y) \in R $. Además $$ (y_{1},f(y_{1})) \in R \ \ \ \ \therefore (y_{1},y) \in R $$
 Lo cual implicaria  que y es cota superior de  de F $ \therefore (w,y) \in R $\\
 y como se tiene además que $ (y,w) \in R $ se concluiria  $ y=w $ lo  cual contradice que $ y \in S \diagdown \lbrace w \rbrace  $ )
 
$$ y_{1}  = y \ \ \ \vee \ \ \ y_{1} \neg y $$

\begin{itemize}
\item[a)] Si $ y_{1} = y $, entonces $ y \in F $ por lo tanto se tiene $ P(y_{1}) $\\
Así como $ y \in S \ \ \ \wedge P(y) $ del \textbf{Hecho 2}\\
dado que $ w \in S $ se tiene 
$$ (w,y) \in R \ \ \ \vee (f(y),w) \in R $$
No ocurre $ (w,y) \in R $ ya que así como $ (y,w) \in R $ \\
Se tendría que $ w=y $ lo cual contradice $ y \in S \diagdown  \lbrace w \rbrace $
$$ \therefore \ \ \ (f(y),w) \in R $$
\item[b)]Suponga ahora que $ y_{1} \neq y $ Como $ y_{1} \in F $ se tiene $ P(y_{1}) $, así como\\
$$ (y, y_{1}) \in R \ \ \ \wedge \ \ y \neq y_{1} \ \ \  y \in S \diagdown \lbrace y_{1} \rbrace  $$
por lo cual se concluye que  $ (f(y), y_{1}) \in R $ y como $ (y_{1}, w) \in R $ se  concluye  $ (f(y),w) \in R $
\end{itemize}
  
\end{itemize}
\textit{i), ii), iii)} muestran que C es u-admisible y por ello $ C=S $
\end{proof}
Sea $ x \in S $, entonces por el \textbf{Hecho 3} se tiene $ P(x) $ \\
Por el \textbf{Hecho 2}, para cualquier $ y \in S $ se tiene $ (y,x) \in R \ \ \ \vee \ \ \  (f(x),y)\in R $\\
Además S es u-admisible así, siendo S una cadena contenida en S, se tiene $ SupS \in S $ \\
Sea $ x_{0} = supS $\\
Como $ f(S) \subset S \ \ \ \therefore f(x_{0}) \in S $
$$(f(x_{0}),x_{0}) \in R \ \ \ \ \wedge \ \ \ \ (x_{0}, f(x_{0})) \in R$$
$$\therefore \ \ \ \ \ x_{0}=f(x_{0})$$

%Fin de la prueba de la proposicion 12
\end{proof} 

\textit{El teorema anterior fue probado por Baurbaki en 1939 y se conoce como el teorema del punto fijo de Baurbaki}

\newpage


\theoremstyle{definition}
\begin{definition}{\textbf{Axioma de elección}}
Sea $ A $ un conjunto cuyos elementos son ajenos no vacios existe un conjunto $ B $  tal que $ \forall \ \ x \in A \diagdown \lbrace \varnothing \rbrace  \ \ \ \ \ B \cap x \ \ \ $ es  un conjunto con un único elemento. 
\end{definition}
\begin{prop13}Son equivalentes las sigueintes proposiciones
\begin{itemize}
\item[I)]Axioma de eleccion
\item[II)]Para cada conjunto $ X $, existe una funcion $$ f: P(X)\diagdown \lbrace \varnothing \rbrace \longrightarrow X $$
Tal que $ f(A) \in A \ \ $ para cada $ A \in P(X) \diagdown \lbrace \varnothing  \rbrace $  
\end{itemize}
\end{prop13}
\begin{proof}
$ I) \rightarrow II) \ \ $ Sea $ X $ un conjunto.\\
Considere al conjunto $ N = \lbrace \lbrace A \rbrace \times A \in P(P(X) \times X) \ \ \vert \ \ A \in P(X) \diagdown \lbrace \varnothing \rbrace \rbrace $\\
El axiomma de especificación permite probar que $ N $  es un conjunto.\\

\bigskip


Se verificara que $ N $ satisface las hipotesis del axioma de elección.\\
Sea $ x \in N  $ entonces $ \exists \ A \in  P(X) \diagdown \lbrace \varnothing \rbrace $ tal que\\
$$ x= \lbrace A \rbrace \times A \ \ \ \ \lbrace A \rbrace \neq \varnothing $$
Así mismo $ A \neq \varnothing \ \ \ \ \therefore x \neq  \varnothing $.\\
Considere otro elemento $y \in N  \ \ \ \  \exists A' \in P(X) \diagdown \lbrace \varnothing \rbrace $ tal que $ y = \lbrace A' \rbrace \times A' $ 
Sea $ t \in  x \cap y $ Como $ t \in y \ \ \ \therefore  t=(A,a) \ \ $ con $ a \in A $ \\
Así mismo como $ t \in y \ \ \  \therefore t=(A',a') \ \ $ con $ a' \in A' $
$$ \therefore \ \ \ (A,a)=(A',a') $$ 
Se tiene en particular que $ A = A' $
$$ x = y$$
Dado que $ N $ satisface las hípotesis del  axioma de elección , existe  un conjunto B tal  que  
$$ \forall \ \ A \in P(X) \diagdown \lbrace \varnothing \rbrace, \ \ \ \ \  B \cap (\lbrace A \rbrace \times A) $$
tiene un único elemento.\\
Sea $ f=B \cap (\cup N) $.\\
$ \ \ u \in f \ \ \ $, entonces $ u \in B \ \ \ \wedge \ \ \ u \in (\cup N) $.\\
Como $ u \in (\cup N) \ \ \ \  \exists m \in N $ tal que $ u \in m \ \ \ $. Por ser $ m $ un elemnto de $ N $ existe  $ A \in P(X) \diagdown \lbrace \varnothing \rbrace \ \ $ tal que  $ m = \lbrace A \rbrace                 \times A  \ \ \ \ \therefore $ como $ u \in m $ existe $ a \in A $ tal que $ u = (A,a)$.\\
$$ \therefore \ \ \ u \in [P(X) \diagdown \lbrace \varnothing \rbrace] \times X $$
Suponga que $ (A,b) \in f $ se verificara que $ a=b $\\
 Como $ (A,b) \in (\cup N) \ \ \ \exists s \in N  $ tal que $(A,b) \in s $\\
 Como $ s \in N   $ existe  $ A' \in P(X) \diagdown \lbrace \varnothing \rbrace $  tal que  $ s= \lbrace A' \rbrace \times A' $
 $$ A=A' \ \ \ \wedge b \in A  $$ 
$\therefore \ \ $ como $ f \subset B $
$$ (A,b) \in B \cap (\lbrace A \rbrace \times A) $$
$$ (A,a) \in B \cap (\lbrace A \rbrace \times A ) $$
Dado que $ b \cup (\lbrace  A \rbrace \times A)  $ tiene un único elemnto, se concluye  que $ a=b $\\
Lo anterior muestra que $ f $ es función.\\
Sea $ k \in P(X) \diagdown \lbrace \varnothing \rbrace  $ se verificara que existe $ w \in X $ tal que 
$$ (k,x) \in f $$
$$ k \subset V  \  \ \wedge  \  \   k \neq \varnothing $$
$$ \therefore \ \ \ \lbrace k \rbrace  \times k \in N $$
$ B \cup ( \lbrace k \rbrace \times k)  \ \ $ Tiene un solo elemento
Sea  $ r $ tal elemnto $ r=(k,w) $ con $ w \in K $\\
Así $ w \in X $,  $ (k,w) \in B $ y $ (k,w) \in cup N $
$$ (k,w) \in f $$

Se probara que el dominio de f es $ P(X) \diagdown \lbrace \varnothing \rbrace $\\
Para concluir que $ f(A) \in A $\\
$ \ \ \forall A  \in P(X) \diagdown \lbrace \varnothing  \rbrace  \ \ \ $ si $ A  \in P(X) \diagdown \lbrace \varnothing  \rbrace \ \ \ $  $ (A,f(A)) \in f $ 
$$  \therefore \ \ \  (A,f(A)) \in \cup N \ \  \ \therefore \  \ \exists \lbrace T \rbrace \times T \in N $$ tal que 
$$ (A,f(A)) \in \lbrace T \rbrace \times T \  \ \ \ \therefore A=T \ \ \ \wedge  \ \ \ f(A) \in T \ \ \therefore f(A) \in A $$

 \newpage
\textit{ii)}$ \Rightarrow $ \textit{i)}\\.
Sea  A un conjunto cuyos elemntos son ajenos no vacios. Se debe probar  que existe un conjunto $ B $ tal que  $ B \cap x $ tiene un único elemnto  $ \forall x \in A $.

Sea $ X = \cup A \ \ $ Por hipotesis existe.
$$ f: P(\cup A ) \diagdown \lbrace \varnothing \rbrace \longrightarrow \cup A $$
Tal que $ \ \ \ f(y) \in y \ \ \  \forall y \in P(\cup A) \diagdown \lbrace \varnothing \rbrace $
Observe que si   $ \ \ y \in A $ entonces  $ \ \ y \in P( \cup A) $\\
(Ya que si $\ \  t \in y \ \  $, entonces $ t \in \cup A \ \ $ y  así  $ \ \ \ y \subset \cup A \ \  $)\\
Además  si $ y \in A \ \  $, $ y \neq \varnothing  $\\
$$ \therefore \ \ \  y \in P(\cup A) \diagdown \lbrace \varnothing \rbrace $$
i.e. $ y $ pertenece al dominio de $ f $ pol lo cual esta definido $ f(y) $  \\
Por el axioma de especificación existe 
$$ B \ \ \forall x  ( x \in B \Leftrightarrow x \in \cup A \ \  \wedge  \ \ \   P(x)) $$
Siendo $P(x)$ la propiedad $ \exists  y \ \ \ x = f(y) $\\
Se verificará que $ B \cap z  $ tiene un único elemento $ \forall z \in A $ \\
Sea  $ z \in A $, entonces $ z \in P(\cup  A) \diagdown \lbrace \varnothing \rbrace  $\\
$ \ \ \ \therefore \ \ $ esta definida $f(z) $\\
Se tiene $ f(z) \in B \ \ \ \wedge  f(z) \in Z \ \ \ \ f(z) \in B \cap Z $\\
Así $ \lbrace f(z) \rbrace  \ \ \subset \ \ B \cap Z $\\
\bigskip
Sea $ v \in B \cap Z $. Como $ v \in B  \ \  \ \  \exists z' \in P(\cup A) \diagdown \lbrace \varnothing  \rbrace $\\
tal que $ v = f(z') $. Como $ f(z') \in Z' $, se tiene $ v \in z' $\\
Además $ \  \ v \in Z \ \ \ \ \ Z' \cap Z \neq \varnothing   \ \ \ \  \therefore z'=z  $ y así.
$$ v=f(z) $$
Lo anterio demuestra que $ B \cap Z \subset \lbrace f(z) \rbrace $
$$ \ \ \therefore \ \ \  B \cap Z = \lbrace f(z) \rbrace  $$

\end{proof} 

\newpage

\begin{prop14}Son equivalentes las siguientes proposiciones.
\begin{itemize}
\item[\textit{i)}]Axioma de elección 
\item[\textit{ii)}](Principio maximal de Hausdorqq)\\
Todo conjunto parcialmente ordenado tiene una cadena maximal, i.e. una cadena que no está contenida propiamente en otra cadena

\item[\textit{iii)}](\textit{Lema de Zorn}) Todo conjunto parcialmente ordenado no vacío,en el cad cadena tiene una cota superior, tiene un elemento maximal
\item[\textit{iv)}](\textit{Principio del buen orden}) Todo conjunto puede ser un buen orden
 
\end{itemize}
\end{prop14}

 \begin{proof}
 \textit{i)} $ \Rightarrow $ \textit{ii)}\\
 Sea $(A,R)$ un conjunto parcialmente ordenado, por el axioma de especificación\\
 Sea $ C= \lbrace x \in P(A) \vert \ \ \ \forall a,b \in x  \ \ \ a < b \ \  \vee \ \  b < a \rbrace $
 $$ \varnothing \in C \ \ \ \  \therefore C \neq  \varnothing $$
 Procediendo por reduccion ela absurdo, suponga que A no tien una cadena maximal.
 
Entoonces $ \forall x \in C $\\
$ \ \ \ \  C_{x} = \lbrace y  \in C \vert x \subset y \wedge x \neq y \rbrace \ \ $ Es  un conjunto no vacío de C
 
Como hipotesis de cumple el Axioma de elección, la proposicion 13 implica, tomsndi $ X=C $, que existe 
$  f:  P(C)  \diagdown  \lbrace \varnothing \rbrace \Longrightarrow C $, tal  que 

$\ \ \ f(u) \in u \ \  \ \forall u \in  P(C) \diagdown \lbrace \varnothing \rbrace  \ \ $ Se tiene así 

$ f(C_{x}) \in C_{x} \ \ \forall  x \in C $\\


Sea $ R_{c} \lbrace (x,y) \in C  \times C \vert \ \ \ x \subset y \rbrace  $
\begin{itemize}
\item[i)]  $ \ \ \ \ (c, R_{c})  $ es un conjunto ordenado
\item[ii)]Considere la finción  $$ g: C \longleftrightarrow C $$ 
$$ x \mapsto f(C_{x})  $$ Observe que\\
$$ (x,g(x)) \in R_{c} $$
$$ x \subset g(x) = f(C_{x}) \in C_{x} $$
Si se verifica que 
\item[iii)] Toda cadena en C tiene un supremo  en C\\
Entonces $ (c,R_{c}) \ $ satisfece las hipotesis  de la proposición 12, por lo cual existe  $ x_{0} \in C$ tal que $ g(x_{0}) = x_{0} $ \\
$\therefore f(C_{x_{0}})=x_{0} \ \ $  Sin embargo $ f(C_{x_{0}}) \in C_{0} $ 
$ \therefore f(C_{x_{0}}) \neq x_{0} $ \\
 Se tendria así un acontradicción.\\

Por lo  tanto para concluir que $ i) \Rightarrow ii) $ basta probar que tod a cadena  en C tiene  en supremo en C.

Sea N  una cadena en C.\\
Se vericará que $ \cup N  $ es supremo de N en C.\\
Sea $ t \in  \cup N $, entonces existe $ z \in N $ tal que $ t \in z $. Como $ z \in N \ \  \vee N \subset C $\\
$ \ \therefore \ \ \ t \in A  $.  Así   $ \cup N \subset A  $ \\
Sean $ a,b \in \cup N $ entonces existen $ u,v \in N $ tales que $ a  \in u \  \ \ \  \wedge
 b \in w $\\
 Como N es una cadena en C se tiene $ u \subset v \vee v \subset u $. \\
  
$ \therefore a, b \in V \ \  \vee  \ \ a,b \in U  $. En cualquier caso como $ u $ y $ v$ son cadenas en A se tiene $ a < b \ \ \vee \ \  b < a $ \\
lo anterior muestra que $ \cup N \in C $\\
\begin{itemize}
\item[a)] Sea $ w \in N $\\
entonces  $ r \in w, \ \ \ r \in \cup N  \therefore w \subset \cup N $ \\
Así $ (w,  \cup N) \in R_{c} \ \ \  \ \forall w  \in N $\\
$  \ \ \ \ \ \  \ \ \ \ \cup N  $ es cota superior de $ N $ \\
\item[b)] Suponga que $ M \in C $ satisface $ (w, M) \in R_{c}  \forall w \in N  $ \\
Entonces $ w \subset  M \ \ \ \  \forall w \in N  $.\\
Sea $ s \in \cup N  $, entonces existe $ m \in N $ tal que $ s \in m $.\\
Como $ m \in N  \  \ \ \ \  (m, M) \in R_{c} $\\
$ \therefore  \  \  \  m \subset M \Rightarrow  s \in M  \  \  \therefore \cup N  $ y así $ (\cup N, M) \in R_{c} $ \\
a),b) muestran que $ \cup N  $ es supremo de $ N $ en $ C $

\end{itemize}

\item[\textit{iii)}$ \Longrightarrow $  \textit{iv)}] Sea $ X $ un conjunto.\\
Se debe probar  que en $  X $ se puede definir un buen orden.

Sea $$ S = \lbrace (A,R) \in P(X) \times P(X \times X) | R \  es \ \  un \   buen \ \  orden \ \  en  \ \   A \rbrace $$
$$ \rho = \lbrace ((A_{1},R_{1}), (A_{2},R_{2})) \in S \times S | A_{1} \subset A_{2} \ \  R_{1} \subset R_{2} \ \ \wedge \ \ [ (x \in A_{1} \wedge y \in A_{2} \diagdown A_{1}) \rightarrow (x,y) \in R_{2} ]  \rbrace $$

Hecho $(S, \rho)$ es un conjunto parcialmente  ordenado no vacío en el que toda cadena tiene una cota superior.\\

\begin{proof}
\begin{itemize}
\item[\textit{i)}] Sea $ (A,R) \in S $
\end{itemize}
\end{proof}


\end{itemize}
 \end{proof}






